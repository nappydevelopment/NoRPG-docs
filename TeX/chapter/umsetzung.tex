\chapter{Umsetzung}

Nachdem im vorherigen Kapitel die technischen Grundlagen erläutert wurden, wird jetzt auf die Umsetzung eingegangen. Dabei wird diese in drei Teile unterteilt

\section{App}
Google Anmeldung wird nicht genutzt, da wir nur ein paar spezielle Informationen von den Spielern brauchen und Google Daten geben??
Vorteil an Android und Google Play Store: Google Play Store: große Anzahl an vielfältigen Apps, Schutz durch Google Play, da Apps Kriterien erfüllen müssen um aufgenommen zu werden

		Durchführen von Usability-Tests

\section{Datenbank auf dem Handy}
Wieso haben wir das gemacht? Vorteile? Nachteile?
Speicherung von Daten --> wieso nicht PlayerPrefs von Unity benutzen sondern Serialisieren? Doku, Paper, ... begründen und Beispiel zeigen. Gespeichert wird in einer .dat anstatt .txt, nicht einfach lesbar bearbeitbar (daten lesen etc.)
	
\section{Datenbank auf dem Server}
Bei der Datenbank auf dem Server handelt es sich um eine MySQL Datenbank. In dieser sind die Daten der Registrierten User und deren Fortschritt in den Tabellen accounts und spielfortschritt gespeichert.

\section{C\# Skripte}

Nachfolgend wird auf die einzelnen C\# Skripte eingegangen, welche für die App benötigt werden. Darüber hinaus wird auf weitere, für die Skripte essentielle umgesetzte Teile der App eingegangen.

\subsection{Player}

Der Player setzt sich aus mehreren einzelnen Objekten zusammen, darunter die Textur für den Spieler, ein Teil deiner Minimap und ein weiteres leeres Objekt, welches für die Kamera wichtig ist. An dem Player selbst befinden sich wiederum die Skripte und von Unity vorgefertigten Objekte. Ein sogenannter Rigedbody sorgt dafür, dass der Player Gravitation erfährt und nicht einfach durch die Luft schweben kann. Ein Character Controller fügt Eigenschaften wie eine Größe und höhe ein. Darüber hinaus besitzt der Player einen Animatoir. Dieser ist zusammen mit dem ThirdPersonController dafür verantwortlich, dass sich der Player in der Szene bewegt und die Animationen korrekt ausgeführt werden. Damit dies passiert, wird diesem Controller der Wert speed und direction übergeben. Daraus resultieren folgene Möglichkeiten der Animation

 \begin{table}[htpb]
 \begin{tabular}{|l|l|l|}
 \hline
  speed & direction & Ergebnis \\
 \hline
  0 & 0 & Animation Idle \\
  >0 & 0 & Animation Walk \\
  >0 & 0.3 & Animation Walk Right Short \\
  >0 & 0.5 & Animation Walk Right Medium \\
  >0 & -0.3 & Animation Walk Left Short \\
  >0 & -0.5 & Animation Walk Left Medium \\
  >0.5 & 0 & Animation Run \\
  >0.5 & 0.3 & Animation Run Right Medium \\
  >0.5 & 0.5 & Animation Run Right Wide \\
  >0.5 & -0.3 & Animation Run Left Medium \\
  >0.5 & -0.5 & Animation Run Left Wide \\ \hline
 \end{tabular}
  \caption{Mögliche Animationen je nach Wert speed und direction}
 \label{tab:tabspeeddirection}
 \end{table}
 
Durch diese vielzahl an Animationen ist gewährleistet, dass der Player zu jeder Zeit die richtige Animation ausführt und die Bewegung nicht unnatürlich aussieht. Die Werte speed und direction für diese Animation kommen dabei aus dem Skript CharacterControll.cs . Diese ist für die Steuerung des Players zuständig. Hier wird die Toucheingabe über den Joystick in Weltkoordinaten umgewandelt und der Player beginnt sich zu bewegen. Dazu ist die Update Methode genutzt worden. In dieser wird, sofern ein Animator an dem Player vorhanden ist, die horizontale und vertikale Bewegung des Joysticks in die direction und den speed umgewandelt. Das ganze passiert dabei in der Methode StickToWorldspace, welche die Transform vom Player, der Kamera, die direction und den speed, übergeben bekommt. Nachdem die Methode aufgerufen wurde und abgeschlossen ist, werden die Werte von direction und speed an den ThridPersonController übergeben und die Animation startet, wie zuvor beschrieben.

In der StickToWorldspace Methode wird zu Begin die rootDirection gesetzt, welche sich dabei aus der Z-Achsen Koordinate zusammensetzt. Anschließend wird die stickDirection gesetzt, durch den horizontalen und vertikalen Wert des Joysticks. Nun wird durch das quadrieren der beiden Werte der speed berechnet. Dieser liegt dabei zwischen Null und Eins.

Danach das ganze in bezug zu der Positionsrichtung der Kamera gesetzt, um die Bewegungsrichtung zu erhalten, um anschließend das Kreutzprodukt aus diesen beiden Werten zu berechen. Mit hilfe dessen kann bestimmt werden, ob sich der Player nach Rechts oder nach Links bewegen soll.

\begin{scriptsize}
\lstset{
	float,
	caption=Skript CharacterController.cs, 
	language=[Sharp]C, 
	frame=single,  
	showstringspaces=false, 
	showspaces=false, 
	numbers=left, 
	captionpos=b, 
	belowcaptionskip=4pt,
	basicstyle=\ttfamily
} 
\newpage
\begin{lstlisting}[label=lst:c_charactercontroller]
using UnityEngine;
using CnControls;
using UnityStandardAssets.CrossPlatformInput;
using UnityEngine.SceneManagement;

public class CharacterControll : MonoBehaviour {

	...

    private float speed = 0.0f;
    private float direction = 0f;
    private float horizontal = 0.0f;
    private float vertical = 0.0f;
    private AnimatorStateInfo stateInfo;

	...	
	
    void FixedUpdate() {
        if(IsInLocomotion() && ((direction>=0 && horizontal>=0) 
        	|| (direction <0 && horizontal < 0))) {
        	
            Vector3 rotationAmount = Vector3.Lerp(Vector3.zero, new Vector3(0f, 
            rotationDegreePerSecound * (horizontal < 0f ? -1f : 1f), 0f), 
            Mathf.Abs(horizontal));
            	
            Quaternion deltaRotation = Quaternion.Euler(rotationAmount * Time.deltaTime);
            this.transform.rotation = (this.transform.rotation * deltaRotation);
        }
    }

    public bool IsInLocomotion() {
        return stateInfo.nameHash == m_LocomotionID;
    }

    void Update() {
    
        if (animator) {
            stateInfo = animator.GetCurrentAnimatorStateInfo(0);
            horizontal = CnInputManager.GetAxis("Horizontal");
            vertical = CnInputManager.GetAxis("Vertical");
            StickToWorldspace(this.transform, gamecam.transform, ref direction, ref speed);
            animator.SetFloat("speed", speed);
            animator.SetFloat("direction", direction, directionDumpTime, Time.deltaTime);
        }
    }
    
    ...

    public void StickToWorldspace(Transform root, Transform camera, 
    	ref float directionOut, ref float speedOut) {
    	
        Vector3 rootDirection = root.forward;
        Vector3 stickDirection = new Vector3(horizontal, 0, vertical);
        speedOut = stickDirection.sqrMagnitude;
        Vector3 CameraDirection = camera.forward;
        CameraDirection.y = 0.0f;
        Quaternion referentialShift = Quaternion.FromToRotation(Vector3.forward, 
        	Vector3.Normalize(CameraDirection));
        	
        Vector3 moveDirection = referentialShift * stickDirection;
        Vector3 axisSign = Vector3.Cross(moveDirection, rootDirection);
        float angleRootToMove = Vector3.Angle(rootDirection, moveDirection) * 
        	(axisSign.y >= 0 ? -1f : 1f);      
        	
        angleRootToMove /= 180f;
        directionOut = angleRootToMove * directionSpeed;
    }
}

\end{lstlisting}
\end{scriptsize}

\subsection{Kamera}





\subsection{Portale}

\subsection{Minimap}

\subsection{Interaktionsmöglichkeiten}
Chests, Händler, etc....

\subsection{Pfadfindungssystem Schiff}



\subsection{Datenimport aus JSON}


\subsection{Datenimport aus / in Datenbank}

Zum Senden der Daten der Registrierung wird das Skript SendDataToServer.cs genutzt. In diesem werden die Daten der Registrierung zwischengespeichert und am Ende an den Server gesendet. Dazu wird die Methode SendRegister() genutzt. In dieser wird die Methode RegisterUser() als Coroutine gestartet. Dazu werden zehn Parameter übergeben, der Username, die Email, das Passwort, der Vorname, der Nachname, das Geburtsdatum, das Geschlecht, der Herkunftsstaat, die native Sprache und der gewählte Charakter. Bei einer Coroutine handelt es sich um einen Thread, welcher beliebig gestarte, pausiert und beendet werden kann.

Innerhalb dieser Methode wird zu Beginn ein Hash erstellt, welcher am Server genutzt wird, um zu überprüfen, ob die Anfrage gültig ist. Dieser besteht dabei aus teilen der Eingabe und einem zusätzlichen geheimen Schlüssel, welchem nur der App und dem Server bekannt sind. Dadurch wird die Sicherheit gesteigert und es wird Angreifern erschwert unberechtigte Zugriffe auf die Datenbank zu tätigen. Bei dm Hash handelt es sich dabei um die MD5 verschlüsselten Eingabedaten. Dadurch ist es fast nicht möglich, einen Validen Hash zu bilden, ohne diese Daten zu kennen.

Nachdem der Hash erstellt wurde, werden alle Parameter in Form einer Url aneinander gehängt. Anschließend wird die URL an ein WWW Objekt übergeben und solange gewartet, bis es eine Antwort gibt. Sofern es keinen Fehler gab, wird ein Text ausgegeben, welche vom Server gesendet wird, andernfalls eine Fehlermeldung.


\begin{scriptsize}
\lstset{
	float,
	caption=Skript SendDataToServer.cs, 
	language=[Sharp]C, 
	frame=single,  
	showstringspaces=false, 
	showspaces=false, 
	numbers=left, 
	captionpos=b, 
	belowcaptionskip=4pt,
	basicstyle=\ttfamily
} 
\newpage
\begin{lstlisting}[label=lst:c_SendDataToServer]
using System;
using System.Collections;
using System.Collections.Generic;
using UnityEngine;
using UnityEngine.UI;

public class SendDataToServer : MonoBehaviour {

    private static string secretKey = "norpg";
    public static string registerURL = "http://norpg.it.dh-karlsruhe.de/register.php?";
    public static string loginURL = "http://norpg.it.dh-karlsruhe.de/login.php?";

	...
	...

    private void SendRegister() {
        StartCoroutine(RegisterUser(userText, emailText, MD5Test.Md5Sum(passwordText), 
        	firstnameText, lastnameText, birthdayText, genderText, 
        	countryText, native_languageText, selected_characterText));
    }

    IEnumerator RegisterUser(string user, string email, string password, string firstname, 
    	string lastname, string birthday, string gender, string country, 
    	string native_language, string selected_character) {

        string hash = MD5Test.Md5Sum(user + email + password 
        	+ firstname + country 
        	+ selected_character + secretKey);

        string post_url = registerURL
            + "user=" + WWW.EscapeURL(user)
            + "&email=" + WWW.EscapeURL(email)
            + "&password=" + WWW.EscapeURL(password)
            + "&firstname=" + WWW.EscapeURL(firstname)
            + "&lastname=" + WWW.EscapeURL(lastname)
            + "&birthday=" + WWW.EscapeURL(birthday)
            + "&gender=" + WWW.EscapeURL(gender)
            + "&country=" + WWW.EscapeURL(country)
            + "&native_language=" + WWW.EscapeURL(native_language)
            + "&selected_character=" + WWW.EscapeURL(selected_character)
            + "&hash=" + hash;
        WWW hs_post = new WWW(post_url);
        yield return hs_post;

        if (hs_post.error != null) {
            print("There was an error posting the high score: " + hs_post.error);
        } else {
            status.text = hs_post.text;
        }
    }
}

\end{lstlisting}
\end{scriptsize}

Auf dem Server läuft für die Datenannahme das PHP Skript register.php. Dieses nimmt die Daten aus der URL entgegen und speichert diese zunächst in Variablen ab. Anschließend wird auch in diesem Skript ein Hash gebildet und anschließend mit dem Mitgesendetem abgegleicht. Sollte es hier einen Fehler geben, sendet der Server einen Error zurück, wenn die Hashes identisch sind, wird eine Verbindung zu der Datenbank aufgabenaut und ein Eintrag in der accounts Tabelle erstellt. Anschließend wird eine erfolgreich Meldung an den Client gesendet.

Für den login innerhalb der App wird identisch vorgegangen. Dabei wird jedoch ein Datensatz in die Datenbank geschrieben, sondern nur gelsen. Des weiteren wird der Hash aus nicht so vielen Werten gebildet, da nur der Username und das verschlüsselte Passwort übermittelt werden. Nachstehend ist die Methode für den Login aus der App und der Code vom Server zur validierug zu sehen.

\begin{scriptsize}
\lstset{
	float,
	caption=Skript SendDataToServer.cs, 
	language=[Sharp]C, 
	frame=single,  
	showstringspaces=false, 
	showspaces=false, 
	numbers=left, 
	captionpos=b, 
	belowcaptionskip=4pt,
	basicstyle=\ttfamily
} 
\newpage
\begin{lstlisting}[label=lst:c_Login]

\end{lstlisting}
\end{scriptsize}

\subsection{Allgemein}
