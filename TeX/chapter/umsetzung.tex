\chapter{Technische Grundlagen}
	Nachstehend werden die technischen Grundlagen erläutert und es wird kurz auf diese eingegangen. Dabei wird mit den Entwicklungsumgebungen begonnen.

\section{Unity3D}

Unity3D ist eine Entwicklungs- und Laufzeitumgebung, die viel in der Spieleindustrie verwendet wird, aber auch in anderen Industrien immer mehr zum Einsatz kommt. Mit diesem Programm können 3D Anwendungen erstellt werden, aber auch 2D oder 2,5D Anwendungen sind möglich. Diese Anwendungen können dank Unity3D plattformübergreifend entwickelt werden.\footnote{Unity3D. Build One Deploy Anywhere. English. Jan. 2016. URL: https://unity3d.
com/unity/multiplatform} Die Entwicklungsumgebung ist dabei an einen 3D-Editor angelegt. Das User Interface (UI) besteht dabei aus verschiedenen Fenstern, um verschiedene Einstellungen der Szenen in Unity3D zu bearbeiten. 

Eines dieser Standardfenster ist das „Scene” Fenster, das in der Mitte von Abbildung \ref{scene} zu sehen ist.\footnote{Unity3D. The Scene View Window. English. Jan. 2016. URL: http://docs.unity3d.com/Manual/UsingTheSceneView.html} In diesem Fenster ist immer die aktuelle Szene dargestellt. Außerdem können Nutzer mit den Objekten aus der Szene interagieren und sie verändern, um sie beispielsweise in der Szene neu platzieren. Sobald ein Objekt in der Szene ausgewählt wurde, öffnen sich in dem Fenster „Inspector” weitere Einstellmöglichkeiten.\footnote{Unity3D. The InspectorWindow. English. Jan. 2016. URL: http://docs.unity3d.com/Manual/UsingTheInspector.html} Diese Einstellmöglichkeiten variieren je nach gewähltem Objekt. Hier können den Objekten zusätzliche Eigenschaften zugewiesen werden, um so ihr Verhalten nochmals zu ändern. Dabei können die visuellen als auch die physischen Eigenschaften der Objekte verändert werden.

Durch das Markieren in dem „Hierarchy” Fenster werden Objekte zusätzlich ausgewählt und es können Eigenschaften zugewiesen werden.\footnote{Unity3D. The HierarchyWindow. English. Jan. 2016. URL: http://docs.unity3d.com/Manual/Hierarchy.html} Dort werden alle Objekte aus der Szene in einer Hierarchie angezeigt. Dabei sind die Objekte auf unterschiedlichen Ebenen dargestellt, um zu zeigen, welches Objekt mit anderen Objekten verknüpft ist. 

Das Fenster „Project” dient dazu, alle Dateien, die in dem Projekt vorhanden sind, anzuzeigen.\footnote{Unity3D. The ProjectWindow. English. Jan. 2016. URL: http://docs.unity3d.com/Manual/ProjectView.html} Außerdem gibt es noch das „Game” Fenster.\footnote{Unity3D. The Game View. English. Jan. 2016. URL: http://docs.unity3d.com/Manual/GameView.html} In diesem Fenster können Ordner, sowie andere Dateien erstellt werden. Oben, in der Mitte der Benutzeroberfläche, befinden sich zudem jeweils ein Start, Pause und Vorlauf Button. Diese Buttons dienen dazu, das Programm im Vorfeld zu testen und in der Entwicklungsumgebung zu rendern. Der Code wird von Unity3D JustIn-Time (JIT) kompiliert, und anschließend auf Mono oder dem Microsoft .NET Framework ausgeführt. Der Code steht in sogenannten Skripten, die in C\#, UnitySkript (ähnlich JavaScript) oder Boo geschrieben sind.

Für den Fall, dass Fehler während des Kompilierens oder während der Laufzeit auftreten, gibt es ein „Console” Fenster. In diesem Fenster können aktuelle Fehlermeldungen ausgegeben werden. Zudem werden hier auch gezielte Meldungen, die in den Skripten programmiert wurden, eingeblendet. Um diesem Fall entgegenzuwirken, gibt es in Unity Tests, die die Szenen auf ihre Korrektheit testen. Die Integrationstests simulieren eine Szene. So können die verschiedenen Objekte auf ihre Eigenschaften geprüft werden. Bei den zuvor in Kapitel 2.1 genannten Prefabs handelt es sich um vorgefertigte Objekte, die in einer Szene verwendet werden können. Dabei sind dies meist Objekte, die mehrmals in einem Projekt verwendet werden können. Bei Skripten handelt es sich wiederum um die Logik, die ein Objekt hat. Kann ein Objekt beispielsweise seine Farbe ändern, wenn ein Nutzer mit diesem Objekt interagiert, steht die Logik dafür in einem Skript. Um diese Logik zu bearbeiten und anzupassen, wird eine geeignete Entwicklungsumgebung benötigt. Bei der Installation von Unity3D ist eine Version von Microsoft Visual Studio enthalten, die zum Bearbeiten von Skripten dient.

\section{Visual Studio}

Microsoft Visual Studio ist eine Entwicklungsumgebung für verschiedene Programmiersprachen.\footnote{Microsoft. Visual Studio-IDE. English. 2015. URL: https://msdn.microsoft.com/de-de/library/dn762121.aspx} Sie wird bei der Unity3D Installation zusätzlich installiert. Mit Visual Studio können die Skripte aus Unity3D bearbeitet werden. Dazu müssen diese Skripte nur mit Visual Studio geöffnet werden. Zusätzlich zu Visual Studio werden auch verschiedene Plug-ins für die IDE installiert. Dabei handelt es sich unter anderem um eine ausführliche Dokumentation von allen in Unity3D zur Verfügung stehenden Methoden und Klassen sowie um Testtools, um verschiedene Tests auszuführen. 

\section{C Sharp}

C\# (gesprochen C Sharp) ist eine von Microsoft entwickelte Programmiersprache, die zusammen mit „.NET 1.0” 2002 in der Version 1 auf den Markt kam und mittlerweile in der Version 6 veröffentlicht ist.\footnote{Microsoft. Visual C#. English. 2015. URL: https://msdn.microsoft.com/de-de/library/kx37x362.aspx} Dabei orientiert sich C\# an den Programmiersprachen Java, C, C++, Haskell und Delphi und nutzt deren grundlegende Konzepte. Aus diesem Grund ist C\# eine objektorientierte Programmiersprache. Die in Unity verwendeten C\# Skripts erben standardmäßig von der Klasse Mono-Behaviour wie in Listing 1 zu sehen. Diese Vererbung sorgt dafür, dass jede Klasse verschiedene Methoden zur Verfügung hat. Dazu zählt eine Start-Methode, die beim Laden eines Objekts mit dem Skript ausgeführt wird und eine Update-Methode, die bei jeder Frameaktualisierung ausgeführt wird. Des Weiteren können weitere Methoden genutzt werden. Außerdem sorgt die MonoBehavior Vererbung dafür, dass diese Skripte mit Objekten in Unity verknüpft werden können.

\section{SQL}

	--> Für die Datenbankverbindung innerhalb der App wichtig , was ist das? Wozu wird es benötight?


\chapter{Umsetzung}

	Nachdem im vorherigen Kapitel die technischen Grundlagen erläutert wurden, wird jetzt auf die Umsetzung eingegangen. Dabei wird diese in drei Teile unterteilt

\section{App}
\section{Datenbank auf dem Handy}
\section{Datenbank auf dem Server}