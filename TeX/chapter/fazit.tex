\chapter{Fazit und Ausblick}
	In diesem letzten Kapitel zurückblickend auf das Projekt, die Probleme und den geplanten Features geschaut. Abgeschlossen wird diese Arbeit mit einem Ausblick. 
	
	\section{Fazit}
	

Bei Hone handelt es sich um eine Spielplattform auf der sich Kinder, bevorzugt aus Regionen in denen Bildung mangelhaft ist, anmelden können. Auf dieser Web-Plattform gibt es für Kinder die Möglichkeit neue Lernspiele für verschiedene Plattformen herunterzuladen. Zusätzlich gibt es eine Ansicht der gelernten Kompetenzen. 

Dieses Konzept hat zwei wesentliche Nachteile für die Benutzer. Für die Verwendung der Web-Plattform wird ein Computer benötigt und gerade weil Spiele für verschiedene Plattformen angeboten werden können, benötigt das Kind mehrere Geräte. Neben diesen Nachteilen, ist das Aussehen und die Bedienung dieser Plattform nicht reizvoll für Kinder gestaltet.

Mehr Kinder als je zuvor in der Geschichte arbeiten täglich mit immer neueren und besseren technischen mobilen Geräten. Deshalb soll eine mobile Applikation, kurz App, für Smartphones entwickelt werden, in der die Kinder auf spielerischer weise Fortschritte machen. Durch die Umsetzung als App wird den Kindern eine Plattform angeboten, mit welcher sie unabhängig und jederzeit auf die Lerninhalte zugreifen können. Ein weiterer Vorteil ist, dass die Kinder den technischen Umgang mit Smartphones lernen. Die App wird unabhängig von Hone funktionieren und es werden keine Inhalte und Funktionen übernommen.

\section{Ziel der Arbeit}

Das Ziel dieses Projektes ist es den Kindern eine Möglichkeit zu geben, mit der sie jederzeit, überall und einfach auf die Lerninhalte zugreifen und spielerisch neues Wissen erwerben können.

Dabei ist es nicht das Ziel die Lerninhalte direkt in der App abzufragen sondern Lernspiele für Smartphones anzubieten, welche es in korrekter Reihenfolge freizuschalten und zu spielen gilt.

Das Ziel dieser Arbeit ist die Vorgehensweise, Bedingungen, Probleme zu dokumentieren. Mit dieser Dokumentation soll gewährleistet werden, dass dieses Projekt von allen Verstanden wird.
	
	
	
	
	
		Noch einmal das Ziel wiederholen 
	
		Projekt allgemein Fazit
		
		Konkrete Probleme: Spiele, Kontrolle über den Fortschritt (wieso deshalb nicht der geplante Fortschrittanzeige entwickelt wurde sondern nur eine Liste, da eine konkrete Überprüfung nicht möglich war)
		
		Fazit aus UI Evaluation - was anders machen
		
		Vielleicht ein bisschen zu komplex für die am anfang abgezielte Altersgruppe - eher für Kinder ab der 5ten Klasse, die schon fliesend lesen können -- ggf. eine Sprachfunktion einbauen, dass die Texte, die angezeigt werden auch vorgelesen werden
		
		

	\section{Ausblick}
		Webfrontend ...
		
		Eigene Spiele zur Überprüfung, ob der Spieler den Standard erfüllt hat...
		
		weitere Klassen und weiter Fächer, dementsprechend auch neue Spielwelten
		
		Ranking/Multiplayer/...