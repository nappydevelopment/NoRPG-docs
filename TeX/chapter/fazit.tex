\chapter{Fazit und Ausblick}

Hone ist eine Spielplattform für Kinder, auf der sie die Möglichkeit haben, Lernspiele herunterzuladen. Die Motivation für ein Lernspiel auf eine Web-Plattform gehen zu müssen ist bei Kindern gering. Darüber hinaus wollen Kinder keine Lernspiele spielen, in denen sie merken, dass sie lernen. Deshalb sollte Hone selbst zu einem Spiel werden, indem sich Kinder anmelden und anschließend spielerisch animiert werden zu lernen. Das ganze sollte als Resultat dieser Arbeit entstehen, dabei waren drei Hauptbestandteile bei der Umsetzung besonders wichtig. 

Kinder sollten eine Möglichkeit bekommen, mit der sie jederzeit, überall und einfach auf die Lerninhalte zugreifen und spielerisch neues Wissen erwerben können. Diese Lerninhalte sollten dabei über externe Lernspiele eingebunden und in korrekter Reihenfolge freigeschaltet werden. Darüber hinaus sollte diese Arbeit als Dokumentation für dieses Projekt dienen, um Vorgehensweisen, Bedienung und Probleme zu dokumentieren. 

Mithilfe der App NoRPG haben Kinder nun eine Möglichkeit überall und zu jeder Zeit spielerisch auf neue Lerninhalte zuzugreifen. Dies ist dadurch realisiert worden, dass die App immer die neusten Standards und externen Lernspiele zu Beginn von einem Server herunterlädt, sofern Internet vorhanden ist, und somit stets aktuell ist. Darüber hinaus können die Kinder auch an verschiedenen Geräten spielen und sind somit unabhängig von Hardware und Ort. 

Die externen Lernspiele werden dabei zentral auf einem Server verwaltet und können in Zukunft über ein Webinterface gepflegt und erweitert werden. Dabei erfüllt jedes Spiel mindestens einen der \acp{CCSS}. Durch eine Prüfung, ob ein Spiel bereits gespielt wurde oder nicht, wird darüber hinaus gewährleistet, dass das Kind auch nur die Spiele angezeigt bekommt, die für den Wissenstand des Kindes geeignet sind. Eine Idee für die Zukunft ist, dass Spiele die sich das Kind herunterlädt, ein Interface implementieren, welches eine Rückmeldung an die App gibt, ob das Kind das vermittelte Wissen erlangt hat. Aktuell wird davon ausgegangen, dass das Kind ein heruntergeladenes Spiel auch spielt und das benötigte Wissen lernt, da es keine Möglichkeit einer Rückmeldung von den externen Spielen gibt.

Gegen Ende der Entwicklung wurde darüber hinaus eine Evaluation des \aclp{UI} durchgeführt. Dabei wurde die Benutzeroberfläche mehrheitlich sehr positiv bewertet. Das konnte durch intuitive Symbole ermöglicht werden, die der Benutzer auch schon aus anderen Spielen und Anwendungen kennt. Durch das intuitive \ac{UI} ist die App für Kinder besonders gut geeignet. Jedoch ist die Registrierung komplex und die Kinder benötigen in diesem Schritt eine Aufsichtsperson zur Hilfe. Darüber hinaus wurde festgestellt, dass die Fortschrittsanzeige für Kinder eher verwirrend ist und sie diese Information nicht benötigen. Als Konsequenz könnte dieser Teil in Zukunft durch ein Webinterface ersetzt werden, auf dem die Eltern den Fortschritt ihres Kindes einsehen könnten. 
	
Ein weiterer Aspekt, der die Komplexität der App für Kinder erhöht, ist die Tatsache, dass viel Textverständnis vorausgesetzt wird. Das Kind sollte bereits flüssig lesen können. Deshalb ist diese Version der App eher für ältere Kinder geeignet. In einem nächsten Schritt kann die App durch eine Text-to-Speach Funktion erweitert werden, damit die Texte vorgelesen werden. Dadurch sinkt die Komplexität und die App ist auch für die gewünschte Zielgruppe einsetzbar.

In der Zukunft sind neben diesen Verbesserungen auch noch weitere Features denkbar. Zum einen sollen in Zukunft auch weitere Schulklassen und Lernfächer angeboten werden, damit die Kinder nicht nur Englisch und Mathematik lernen können. Dadurch wird die App noch attraktiver und mehr Wissen kann an die Kinder vermittelt werden. 

Um das Wissen der Kinder innerhalb der App zu testen, wäre es möglich, eigene Spiele in der App anzubieten. Diese Spiele könnten dann wie eine Art Test angesehen werden, bei der gelerntes Wissen überprüft wird. Außerdem könnten in Zukunft ein Multiplayer oder ein Ranking eingeführt werden, um einen direkten Vergleich mit beispielsweise Freunden zu haben und somit eine weitere Motivation zu schaffen.

Zusammenfassend kann gesagt werden, dass die App ein guter Schritt in die richtige Richtung ist, um Lernspiele sinnvoller zu nutzen und Kinder effektiv zu animieren etwas lernen zu wollen, ohne das sie es merken.