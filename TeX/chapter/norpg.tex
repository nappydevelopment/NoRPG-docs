\chapter{NoRPG}
Bei NoRPG handelt es sich um ein Spiel, welches an ein Role Player Game (RPG) erinnern soll. Bei RPGs spielt der Spieler einen Chracter in einer fiktiven Welt. Dabei gibt es eine Story die wärend des spielens erlebt wird. Damit diese Geschichte vorran geht, muss der Spiele verschiedene Quests, dabei handelt es sich um verschiedenste Aufgaben, erledigen, um einen Fortschritt im Spiel zu erlangen. Darüber hinaus sammelt der Spieler Objekte in der Welt die er im Spiel nutzen kann. Ein Beispiel für solche Spiele ist das Spiel The Witcher 3 welches auf diese Prinzipien aufbaut. 
	
Darüber hinaus gibt es noch sogenannte MMORPGs, Massive Multiplayer Online Role Player Game. Dabei Gibt es wie bei RPGs eine Story und Quests allerdings kann man auch auf andere Spieler treffen und mit diesen zusammen spielen. Das wohl berühmteste MMORPG ist dabei World of Warcraft vom entwickler Blizzard. Hier kann der Spieler zwischen verschiedenen Klassen einen Chrktere auswählen, von Menschenähnlichen bis zu komischen Gestalten. Jede dieser Klassen hat verschiedene Fähigkeiten welche sich auf den Spieleverlauf auswirken. Darüber hinaus gibt es in MMORPGs verschiedene Events an denen die Spieler gemeinsam versuchen eine Quest zu erfüllen. Dieser Multiplayer grenzt die MMORPGs von den RPG ab.

NoRPG hat keine Multiplayer möglichkeiten und ist deswegen nur ein RPG. Allerdings muss der Spieler keine Quests erfüllen wie sie aus anderen Spielen bekannt sind. Der Nutzer muss andere Spiele spielen, damit die Geschichte im Spiel vorran geht. Dshalb wurde sich für den Namen NoRPG entschieden, da es sich bei dem Spiel um ein RPG handelt, aber nicht alle klassischen Eigenschaften eines RPGs besitzt.
	
NoRPG ist ein Spiel welches versucht Bildung für jeden erreichbar zu machen. Dieses Ziel ist dabei in den Global Goals definiert. Dabei handelt es sich um 17 Ziele welche bis 2030 Umgesetzt werden sollen um das Leben für alle Menschen auf der Welt zu verbessern. 2015 haben 193 Weltführer diese Unterzeichnet und begonnen diese Umzusetzen. Dabei sind diese Ziele umfangreich und reichen von einem besseren Umgang mit den uns zur Verfügung stehenden Ressourcen bis hin zu qualitativ hochwertiger Bildung für jeden und kostenlos.

NoRPG unterstützt dabei das Ziel Hochwertige Bildung für jeden Zugänglich zu machen, da die App für jeden frei zugänglich ist. Dieses Ziel hat weitere Unterziele, wobei nun kurz auf die für NoRPG relevanten Unterziele eingegangen wird.

\begin{itemize}
\item Bis 2030 sicherstellen, dass alle Mädchen und Jungen gleichberechtigt eine kostenlose und hochwertige Grund- und Sekundarschulbildung abschließen, die zu brauchbaren und effektiven Lernergebnissen führt.

Dieses Unterziel wird in NoRPG dahingegend unterstützt, dass die Common Core State Standards implementiert. Dies sind standartisierte Standarts für Unterrichtsfächer und beschreiben den zu erlernenden Inhalt für Kinder in den verschiedenen Klassen. Da NoRPG für alle kostenfrei zugänglich ist und Mädchen und Jungen gleichberechtigt sind werden auch diese zwei Aspekte des Unterziels unterstützt.

\item Aufbau und Weiterentwicklung von Bildungseinrichtungen, die kinder- und behindertengerecht und geschlechtsspezifisch sind und für alle eine sichere, gewaltfreie, integrative und effektive Lernumgebung bieten

Da NoRPG keine Bildungseinrichtung ist wird dieses Unterziel nur bedingt erfüllt. NoRPG bietet Kindern jedoch eine sichere, gewaltfreie, integrative und effektive Lernumgebung, wochurch dieses Ziel zum Teil erfüllt wird. Darüber hinaus ist diese Umgebung für Kinder ausgelegt und somit kindergerecht. Geschlechtsspezifisch ist das Spiel nur dahingehend, dass die Kinder zu begin einen Charaktere in ihrem Geschlecht und Alter auswählen können.

\end{itemize}

	\url{http://www.globalgoals.org/de/global-goals/quality-education/}

	
\section{Gedanken}
	
	GAMIFICATION -- was ist gamification wieso wird das gemacht, hier rein passt nicht bei SRS
	
\section{Story}

In NoRPG spielt der Spieler einen Charakere der zu Begin der Geschichte in einem Dorf wohnt. Von diesem können die Einwohner durch Portale in 5 andere Welten gehen, in denen andere Menschen leben und andere Lebensbedingungen herschen. Eines Tages wurde dieses Dorf von einem Bösewicht heimgesucht und dieser hat die komplette Welt, in der der Spieler lebt, farblos gemacht. Zusammen mit den Farben wurden die Emotionen aus dem Dorf genommen. Nun hat sich der Hauptcharaktere das Ziel gesetzt, die Farben und die Emotionen der Einwohner zurück zu bringen, damit wieder alles wie vor dem Angriff ist. Dazu muss der Spieler in die verschiedenen Welten gehen und verschiedene Aufgaben erfüllen.

In Welt 1, einer Waltwelt, muss der Spieler verschiedene Edelsteine sammeln und Aufgaben erfüllen, um die ersten Farben zurück zu erlangen. Sobald dies geschehen ist kommt die Farbe Grün in das Dorf des Spielers zurück. Nun muss der Spieler in die anderen Welten gehen und die Farbe zurück holen. Über dem Dorf wird sein aktueller Fortschritt als Regenbogen dargestellt, welcher umso bunter und voller wird, unso mehr Edelsteine gefunden und Aufgaben abgeschlossen wurden. Sobald der Spieler alle Edelsteine gefunden und alle Aufgaben abgeschlossen hat, hat er die Welt gerettet und es kommt eine Endsequenz.
	
\section{Abgrenzung zu MOOC}

NoRPG ist allerdings kein Massive Open Online Course (MOOC) sondern baut nur auf einem auf. Die Daten dieses MOOCs werden in NoRPG zur verfügung gestellt, damit die Kinder einen Anreiz haben, diesen zu nutzen. Der eigentliche MOOC ist dabei Hone. Dabei handelt es sich um eine Lernplattform, welche es für Kinder ermöglicht, grundlegendes Verständniss, welches in der Grundschule vermittelt wird, zu erlangen. Dabei stellt Hone den Kindern eine Liste von Spielen bereit. Durch spielen dieser Spiele lernen die Kinder die Dinge die sie auch in der Grundschule lernen würden. Dabei sind diese Lerninhalte in den CCSS definiert und aufgegliedert.