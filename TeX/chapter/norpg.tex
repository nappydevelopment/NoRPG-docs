\chapter{NoRPG}		
Bei NoRPG handelt es sich um eine Gamifizierung einer Lernspielplattform. Dabei soll NoRPG an ein Rollenspiel, bzw. Role Player Game (RPG), erinnern und implementiert dessen charakteristische Eigenschaften. Eine dieser Eigenschaften von RPGs ist, dass der Spielende in die Rolle realer Menschen, fiktiver Figuren, Tiere oder auch Gegenstände einer fiktiven Welt schlüpft\footnote{vgl. Warwitz und Rudolf \cite{rpgSinn} Seite 78ff.}. Dabei wird eine Story erzählt, die das Kind erleben kann. Damit der Spieler allerdings in der Geschichte vorankommt, muss er verschiedene Missionen bzw. Quests erledigen. Dabei kann es sich um die verschiedensten Aufgaben handeln. Darüber hinaus sammelt der Spieler Objekte in der Welt, welche er anschließend im Spiel nutzen kann. Ein Beispiel für solche Spiele ist das Spiel The Witcher 3, welches auf diesen Prinzipien aufbaut\footnote{für weitere Informationen siehe \url{http://thewitcher.com/en/witcher3}}.
	
Des Weiteren gibt es noch MMORPGs, Massive Multiplayer Online Role Player Game. Dabei gibt es wie bei RPGs eine Story und Quests, allerdings kann man auch auf andere Spieler treffen und mit ihnen gemeinsam spielen. Das wohl berühmteste MMORPG ist dabei World of Warcraft \footnote{für weitere Informationen siehe \url{https://worldofwarcraft.com/}} vom Entwickler Blizzard. In MMORPGs gibt es die Möglichkeit verschiedene Events mit mehreren Spieler gemeinsam Quests zu erfüllen. Dieser Mehrspieler-Modus grenzt die MMORPGs von den RPGs ab.

Jedoch handelt es sich bei NoRPG letztendlich nicht um ein klassisches RPG oder MMORPG, sondern um eine Lernspielplattform und bietet Lernspiele zum Herunterladen an. NoRPG soll durch die Eigenschaften eines Rollenspiels die Spieler dazu anregen, weitere Lernspiele herunterzuladen und zu spielen. Deshalb wurde sich für den Namen NoRPG entschieden, da nicht alle charakteristischen Eigenschaften implementiert werden.

NoRPG ist allerdings auch nicht mit einem Massive Open Online Course (MOOC) zu verwechseln, sondern baut nur auf einem auf. Ein MOOC bezeichnet einen kostenlosen Onlinekurs. Ein MOOC würde selbst die Lerninhalte anbieten\footnote{Vgl. Porter \cite{moocBook} Seite 3f.}, wohingegen in NoRPG nur Spiele angeboten werden, die den Lerninhalt bereitstellen.

\section{The Global Goals}
NoRPG ist ein Spiel, welches versucht Bildung für jeden erreichbar zu machen. Dieses Ziel ist dabei in den Global Goals definiert. Dabei handelt es sich um 17 Ziele welche bis 2030 Umgesetzt werden sollen, um das Leben für alle Menschen auf der Welt zu verbessern\footnote{vgl. Global Goals \cite{global} (2017)}. 2015 haben 193 Weltführer diese Unterzeichnet und begonnen dieses umzusetzen. Dabei sind diese Ziele umfangreich und reichen von einem besseren Umgang mit den uns zur Verfügung stehenden Ressourcen bis hin zu qualitativ hochwertiger Bildung für jeden und kostenlos.

NoRPG unterstützt dabei das Ziel, hochwertige Bildung für jeden zugänglich zu machen. Dieses Ziel wird beschrieben als Sicherung eines integrierenden Bildungssystems für alle und die Förderung von gleichberechtigten und hochwertigen lebenslangen Lernchancen\footnote{\url{http://www.globalgoals.org/de/global-goals/quality-education/}}. Dieses Ziel hat weitere Unterziele, wobei nun kurz auf die für NoRPG relevanten Unterziele eingegangen wird.

\begin{itemize}
\item Bis 2030 sicherstellen, dass alle Mädchen und Jungen gleichberechtigt eine kostenlose und hochwertige Grund- und Sekundarschulbildung abschließen, die zu brauchbaren und effektiven Lernergebnissen führt.

\item Aufbau und Weiterentwicklung von Bildungseinrichtungen, die kinder- und behindertengerecht und geschlechtsspezifisch sind und für alle eine sichere, gewaltfreie, integrative und effektive Lernumgebung bieten.
\end{itemize}

Das erste Unterziel wird in NoRPG dahingegen unterstützt, dass die Common Core State Standards implementiert werden. Da NoRPG für alle kostenfrei zugänglich ist und Mädchen und Jungen gleichberechtigt sind, werden auch diese zwei Aspekte des Unterziels unterstützt. 

Da NoRPG keine Bildungseinrichtung ist wird das zweite Unterziel nur bedingt erfüllt. NoRPG bietet Kindern jedoch eine sichere, gewaltfreie, integrative und effektive Lernumgebung, wodurch dieses Ziel allerdings zum Teil erfüllt wird. Darüber hinaus kann diese Anwendung in Bildungseinrichtungen angewendet werden. Geschlechtsspezifisch ist das Spiel nur dahingehend, dass die Kinder zu Beginn das Geschlecht ihres Charakters auswählen.

\section{Common Core State Standards}
Die Common Core sind ein Set von hochqualitativen akademischen Standards für Mathe und Englisch. Diese Lernziele skizzieren, was ein Schüler wissen und fähig sein am Ende jeder Klasse sollte. Die Standards wurden geschaffen um sicherzustellen, dass alle Schüler mit den gelernten Fähigkeiten und Kenntnissen im Collage, Karriere und im Leben, egal wo sie leben, erfolgreich zu sein.

Das Problem des Schulsystems ist, dass die Standards von Staat zu Staat variieren und stimmen meistens nicht mit dem überein, was die Kinder wissen sollten. In Erkennung der Notwendigkeit konsequenter Lernziele wurden die Common Core State Standards entwickelt\footnote{Vgl. \url{http://www.corestandards.org/about-the-standards/}}.

Die Standards werden dabei von den zur Verfügung gestellten Spielen erfüllt. NoRPG sorgt dafür, dass diese Standards in der korrekten Reihenfolge und vollständig erfüllt werden. Betrachtet werden zunächst nur die Standards für die ersten fünf Klassen. Die Abarbeitung der Standards entspricht nicht dem klassischen Vorgehen von Schulen, so ist es beispielsweise nicht notwendig alle Standards der ersten Klasse zu erfüllen, um Stoff der zweiten Klasse freizuschalten. So ist es möglich, wenn ein Kind den Standard Geometrie der ersten Klasse vollständig erfüllt, schon weiter mit Geometrie für die zweite Klasse fortzufahren, ohne das alle anderen Mathestandards der ersten Klasse vollständig abgeschlossen wurden.

\subsubsection{Mathe}
Die Common Core State Standards konzentrieren sich auf eine klare Reihe von mathematischen Fähigkeiten und Konzepten. Die Schüler lernen Konzepte in einer organisierten Weise. Die Standards ermutigen die Schüler, reale Probleme zu lösen\footnote{Vgl. \url{http://www.corestandards.org/Math/}}.

Das Fach Mathe lässt sich für die ersten fünf Klassen in insgesamt fünf Themenbereiche eingliedern. Zu den Themen gehören beispielsweise algebraisches Denken, Operationen im Zahlenraum bis 100, Maßeinheiten und Geometrie. Jedes dieser einzelnen Themen sind nochmals in Standards unterteilt. Erst durch diese genaue Gliederung wird es ermöglicht, dass alle wichtigen Inhalte abgedeckt werden. Damit das vorhin beschriebe Konzept umgesetzt werden kann, mussten sehr früh alle Abhängigkeiten zwischen den Standards ermittelt und visualisiert werden. Der vollständige Abhängigkeitsbaum ist im Anhang dieser Arbeit zu finden.

\subsubsection{Englisch}
Die Common Core State Standards bittet die Schüler, Geschichten und Literatur zu lesen, sowie komplexere Texte, die Fakten und Hintergrundwissen in Bereichen wie Wissenschaft und Sozialwissenschaften liefern. Die Schüler werden herausgefordert und gefragt, welche sie dazu bringen, auf das zurückzugreifen, was sie gelesen haben. Dies unterstreicht kritisches denken, Problemlösung und analytische Fähigkeiten, die für den Erfolg in College, Karriere und Leben erforderlich sind\footnote{Vgl. \url{http://www.corestandards.org/ELA-Literacy/}}.

Das Fach Englisch lässt sich für die ersten fünf Klassen in insgesamt sechs Themenbereiche eingliedern. Zu den Themen gehört Lesen von unterschiedlichen Texten, Schreiben, Sprechen, Zuhören sowie Grammatik. Die Gliederung diese Themen entspricht den Standards. Auch hier ist der Abhängigkeitsbaum im Anhang dieser Arbeit zu finden.

\section{Gamifizierung}
Dieses Ziel der Global Goals soll dabei durch die Gamifizierung der Common Core State Standards umgesetzt werden. Gamifizierung bzw. Gamification bezieht sich auf die Analyse von spielespezifischen Eigenschaften, welche die Spiele unterhaltsam machen und diese dann in Situationen außerhalb von Spielen anzuwenden, um das Gefühl von Spaß für neue Anwendungen, wie Lernen oder Marketing, zu übertragen\footnote{Umformuliert vom Oxford Dictionary}. Ein Beispiel dafür ist PayBack. Dabei sammeln die Nutzer bei jedem Einkauf Punkte. Diese können die Kunden dann gegen Prämien eintauschen. In Videospielen sammeln die Spieler zum Beispiel Münzen um diese anschließend gegen Gegenstände einzutauschen. 

Gamifizierung verwendet darüber hinaus noch weitere erfolgreiche Prinzipien aus Videospielen, um die Nutzer zu Motivieren das Produkt zu nutzen. Eines der wichtigsten Prinzipien ist die Einbettung der Handlungen in eine Geschichte. Die Erzählung einer gut durchdachten Geschichte hilft dem Spieler sich in die Rolle seines Charakters zu versetzen und eine emotionale Bindung herzustellen. Dadurch wird das Spielerlebnis intensiver und macht dem Spieler auch mehr Spaß. Die Wichtigkeit von Spaß ist dabei nicht zu unterschätzen. Spaß motiviert die Kinder, auf eigene Faust zu lernen und das Spiel mehr zu spielen\footnote{Vgl. Michael und Chen \cite{seriousGamesFun} Seite 40}.

Die Möglichkeit der Wahl ist ein weitere wichtiges Element von Spielen. Spiele wären nicht spaßig ohne die Auswahl. Tatsächlich würde es sich um Filme oder Bücher handeln. Diese Medien haben keine Wahlmöglichkeiten, entweder der Betrachter ist Aufmerksam oder nicht. Die Wahl, in Bezug auf die Handhabung einer bestimmten Herausforderung als aktiver Teilnehmer, Kontrolle über das eigene Lernen zu haben, macht die eigenen Entscheidungen und deren Konsequenzen bewusst\footnote{Vgl. Routledge \cite{seriousGamesPrinciples} Seite 30f.}.

Ein weiteres Prinzip welches oft verwendet wird ist direktes Feedback. Dieses wird in den meisten Fällen zusammen mit Belohnungen verwendet. Es ist schwer sich das Leben ohne jegliche Rückmeldung vorzustellen, in der Tat ist es unmöglich. Rückmeldungen sind Informationen welche den Menschen helfen die Welt um uns herum besser zu verstehen und dies ist entscheidend für die Weiterentwicklung. Durch zusätzliche Belohnungen für besondere bzw. korrekte Leistungen verstärkt sich dieser Effekt. Der Spieler weiß etwas richtig gemacht zu haben\footnote{Vgl. Routledge \cite{seriousGamesPrinciples} Seite 32ff.}. Allerdings ist zu beachten, dass die Belohnungen auch einen Nutzen für den Spieler haben, damit der Ehrgeiz besteht diese Belohnungen zu sammeln\footnote{Vgl. Berkling, Faller und Piertzik \cite{gamesPaper} Seite 8}.

Status durch Level und Auszeichnungen

In Onlinespielen sind die Prinzipien Wettbewerb und Teamaktivität einer der wichtigsten. Messen mit anderen und zusammen mit anderen erleben. Darüber hinaus sind die Faktoren wie Erfolgserlebnisse, Gruppenzugehörigkeit und soziale Akzeptanz wichtig. 

Im nächsten Unterkapitel wird das Prinzip der Einbettung in eine Geschichte näher erläutert. Dies erfolgt ausführlicher, um die Gründe zu verstehen, wieso etwas so wie es ist implementiert wurde.
	
\section{Die Geschichte}
In NoRPG spielt der Spieler einen Charakter, der in dem Dorf Rutherglen wohnt. Am Anfang der Geschichte wird dieses Dorf von dem Bösewichten heimgesucht. Dieser klaut alle Farben das komplette Dorfes und alles wirkt farblos und somit auch emotionslos. Nun hat der Spieler, das Ziel die Farben wieder zurück zu bringen. Dazu muss der Spieler in verschiedenen Welten gehen und verschiedene Quests erfüllen.

Die einzelnen Welten repräsentieren eine Klassenstufe und sind dementsprechend anspruchsvoll und unterschiedlich gestaltet. Die Standards der ersten Klasse sind in der ersten Welt. Äquivalent verhält es sich auch in den anderen Welten. Neben den Standards gibt es noch Truhen zu finden, die den Spieler in der Geschichte von NoRPG voranbringen. Jede Welt ist dabei in Unterbereiche gegliedert, die der Spieler mit der Zeit erreichen kann. Dabei wird der zu erkundende Bereich immer größer, je weiter der Spieler in der Geschichte vorankommt. 

Eine sehr markante Eigenschaft von RPGs ist, dass die Spielwelt offen gestaltet ist und dem Spieler nur eine ganz grobe Vorgehensweise vorgeben wird. Der Spieler kann sich frei in der jeder Welt bewegen, kann sich mit allen Bewohnern unterhalten und verschiedene Sachen, wie beispielsweise Ruinen, etc., entdecken. Dies wird als Open World bezeichnet. Dem Spieler werden nur Grenzen vorgegeben in denen das Kind seine eigene Herangehensweise entwickeln kann. Eine grobe Vorgabe ist notwendig, damit das lernen strukturiert stattfindet.

\subsubsection{Startwelt: Das Dorf Rutherglen}
	Rutherglen ist die Heimat des Spielers und dient als Brücke zwischen allen Welten. In die verschiedenen Welten gelangt der Spieler über Portale, die in Rutherglen verteilt sind. Bei dem Heimartort des Spielers handelt es sich um ein verschlagenes, unscheinbares und ruhiges Dorf. Der Spieler kann in dieser Welt mit allen Bewohnern interagieren und sprechen. Diese erzählen Geschichten über das Dorf, über ihr Leben oder geben Tipps und Hinweise.
	
\subsubsection{Welt 1: Die dichten Wälder Talhan}
	Talhan ist das Waldgebiet von Rutherglen. In dieser Welt ist der Wald das primäre Element. Dieser ist in verschiedenen Ausprägungen vorhanden. Das Kind findet sich in einem großen Wald mit verschiedenen Baumarten wieder: Von Laubbäumen, über tropische Bäume bis hin zu großen Fichtenbäume. Neben diesen Elementen gibt es noch weitere Sachen zu entdecken, die die Reise des Kindes durch die Wälder Talhan spannender gestalten soll. Bei Abschluss dieser Welt kehrt auch die erste Farben wieder zurück in das Dorf. Dabei handelt es sich um die Farbe Grün.
	
\subsubsection{Welt 2: Die tropischen Inseln von Galapagos}
	Bei der nächsten Welt handelt es sich um die tropischen Inseln von Galapagos. Diese Welt ist im Gegensatz zum Wald sehr farbenfroh, allerdings gibt es auch viel Wasser. Galapagos besteht dabei aus mehreren Inseln, welche mit dem Schiff erreicht werden können. Auch hier gibt es auf den verschiedenen Inseln unterschiedliche Sachen zu entdecken. Bei Abschluss dieser Welt kehrt nun auch die nächste Farbe wieder zurück in das Dorf. Dabei handelt es sich um die Farbe Blau.
	
\subsubsection{Welt 3: Die endlose Wüste Kalahari}
	Kalahari ist eine sehr große Wüste und ist trostloser wie die vorherigen Welten. In dieser Welt wird der Begriff Open World sehr deutlich. Im Vergleich zu fast allen anderen Welten gibt es keinen Weg oder Wegweiser. Dieses Open World Konzept hat den Sinn, das Gefühl einer endlosen riesigen Wüste zu transportieren. Innerhalb dieser Wüste kann das Kind Oasen, verschiedene Ruinen und ägyptische Wahrzeichen, wie beispielsweise Pyramiden, entdecken. Auch hier erhält der Spieler bei Abschluss die nächste Farbe. Dabei handelt es sich um die Farbe Gelb.
	
\subsubsection{Welt 4: Das verschneite Gebirge Lhotse}
	Die vorletzte und größte Welt sind die verschneite Gebirge Lhotse. In dieser Eiswelt dreht sich alles um Eis und Schnee. Neben der Wüste ist diese Welt die zweite Welt, in der das Prinzip Open World sehr deutlich wird. Es gibt viele hohe Berge, verschiedene große Höhlen und weitere Sachen. Nachdem das Kind alle Truhen auch in dieser Welt gefunden hat erhält es die vorletzte Farbe Türkis.
	
\subsubsection{Welt 5: Der Vulkan Ätna}
	Die zunächst letzte Welt ist Ätna. Hier dreht sich alles um Feuer. In dieser Welt gibt es verschiedene Inselplattformen, welche durch Lava getrennt sind. Allerdings sind einige untereinander mit Brücken verbunden. Auf den verschiedenen Inseln kann der Spieler verschiedene Dinge erkunden, darunter Drachen, Vulkane oder Ruinen. Nach dem das Kind auch diese Welt bewältigt, sind alle Farben wieder zurück im Dorf. Aus dieser Welt bekommt das Kind die fehlende letzte Farbe Rot.
