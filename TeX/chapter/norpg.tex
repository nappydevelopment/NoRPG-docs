\chapter{NoRPG}
Bei NoRPG handelt es sich um ein Spiel, welches an ein Role Player Game (RPG) erinnern soll. Bei RPGs spielt der Spieler einen Chracter in einer fiktiven Welt. Dabei gibt es eine Story die wärend des spielens erlebt wird. Damit diese Geschichte vorran geht, muss der Spiele verschiedene Quests, dabei handelt es sich um verschiedenste Aufgaben, erledigen, um einen Fortschritt im Spiel zu erlangen. Darüber hinaus sammelt der Spieler Objekte in der Welt die er im Spiel nutzen kann. Ein Beispiel für solche Spiele ist das Spiel The Witcher 3 welches auf diese Prinzipien aufbaut. 
	
Darüber hinaus gibt es noch sogenannte MMORPGs, Massive Multiplayer Online Role Player Game. Dabei Gibt es wie bei RPGs eine Story und Quests allerdings kann man auch auf andere Spieler treffen und mit diesen zusammen spielen. Das wohl berühmteste MMORPG ist dabei World of Warcraft vom entwickler Blizzard. Hier kann der Spieler zwischen verschiedenen Klassen einen Chrktere auswählen, von Menschenähnlichen bis zu komischen Gestalten. Jede dieser Klassen hat verschiedene Fähigkeiten welche sich auf den Spieleverlauf auswirken. Darüber hinaus gibt es in MMORPGs verschiedene Events an denen die Spieler gemeinsam versuchen eine Quest zu erfüllen. Dieser Multiplayer grenzt die MMORPGs von den RPG ab.

NoRPG hat keine Multiplayer möglichkeiten und ist deswegen nur ein RPG. Allerdings muss der Spieler keine Quests erfüllen wie sie aus anderen Spielen bekannt sind. Der Nutzer muss andere Spiele spielen, damit die Geschichte im Spiel vorran geht. Dshalb wurde sich für den Namen NoRPG entschieden, da es sich bei dem Spiel um ein RPG handelt, aber nicht alle klassischen Eigenschaften eines RPGs besitzt.
	
NoRPG ist ein Spiel welches versucht Bildung für jeden erreichbar zu machen. Dieses Ziel ist dabei in den Global Goals definiert. Dabei handelt es sich um 17 Ziele welche bis 2030 Umgesetzt werden sollen um das Leben für alle Menschen auf der Welt zu verbessern. 2015 haben 193 Weltführer diese Unterzeichnet und begonnen diese Umzusetzen. 

NoRPG unterstützt dabei das Ziel Hochwertige Bildung für jeden Zugänglich zu machen, da die App für jeden frei zugänglich ist. Dieses Ziel  hat weitere Unterziele, wobei nun kurz auf die für NoRPG relevanten Unterziele eingegangen wird.

\begin{itemize}
\item Bis 2030 sicherstellen, dass alle Mädchen und Jungen gleichberechtigt eine kostenlose und hochwertige Grund- und Sekundarschulbildung abschließen, die zu brauchbaren und effektiven Lernergebnissen führt
\item Bis 2030 sicherstellen, dass alle Mädchen und Jungen Zugang zu hochwertiger frühkindlicher Erziehung, Betreuung und Vorschulbildung erhalten, damit sie auf die Grundschule vorbereitet sind
\item Bis 2030 sicherstellen, dass alle Lernenden die notwendigen Kenntnisse und Qualifikationen zur Förderung nachhaltiger Entwicklung erwerben, unter anderem durch Bildung für nachhaltige Entwicklung und nachhaltige Lebensweisen, Menschenrechte, Geschlechtergleichstellung, eine Kultur des Friedens und der Gewaltlosigkeit, Weltbürgerschaft und die Wertschätzung kultureller Vielfalt und des Beitrags der Kultur zu nachhaltiger Entwicklung
\item Aufbau und Weiterentwicklung von Bildungseinrichtungen, die kinder- und behindertengerecht und geschlechtsspezifisch sind und für alle eine sichere, gewaltfreie, integrative und effektive Lernumgebung bieten
\end{itemize}



	
	NoRPG unterstützt die Global Goals, 4. Ziel Quality Education. Paar Worte dazu verlieren. 
	
	\url{http://www.globalgoals.org/de/global-goals/quality-education/}

	
\section{Gedanken}
	
	GAMIFICATION -- was ist gamification wieso wird das gemacht, hier rein passt nicht bei SRS
	
\section{Story}
	
\section{Abgrenzung zu MOOC}
	Was ist ein MOOC. Wieso ist unser kein klassischen MOOC.