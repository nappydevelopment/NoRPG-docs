\chapter{NoRPG}

%%%%DAS WAR MEINE BESCHREIBUNG IM SRS VIELLEICHT WILLST DU WAS ÜBERNEHMEN WIE DIE QUELLE OR STH ENJOY

%Bei NoRPG handelt es sich um eine Gamifizierung einer Lernspielplattform für Android. NoRPG stellt unterschiedlichste Lernspiele bereit und ist dabei wie ein Rollenspiel aufgebaut und besitzt auch dieselben charakteristischen Eigenschaften eines Rollenspiels. Eine charakteristische Eigenschaft von Rollenspielen ist, dass der Spielende in die Rolle realer Menschen, fiktiver Figuren, Tiere oder auch Gegenstände übernimmt\footnote{vgl. Warwitz & Rudolf \cite{rpgSinn} Seite 78ff.}. 
		
%Jedoch handelt es sich bei NoRPG letztendlich nicht um ein klassisches Rollenspiel sondern immer noch um eine Lernspielplattform und bietet Lernspiele zum Herunterladen an. NoRPG soll durch die Eigenschaften eines Rollenspiels die Spieler dazu anregen, weitere Lernspiele herunterzuladen und zu spielen. Dabei wird die Software die Lernspiele in einer festgelegten Reihenfolge anbieten und mit Hilfe einer Anzeige, den aktuellen Fortschritt des Spielers festhalten.	
		%%%%%%%%%%%%%%%%%%%%%%%%%%%%%%%%%%%%%%%%%%%%%%%%%%%%%%%%%%%%%%%%%%%%%%%%%
		
%Bei NoRPG handelt es sich um eine Gamifizierung einer Lernspielplattform für Android.  <-- Vielleicht nmoch ein allgemeiner satz davor bevor du in das geschehen reingehst :D
Bei NoRPG handelt es sich um ein Spiel, welches an ein Role Player Game (RPG) erinnern soll. Bei RPGs spielt der Spieler einen Charakter in einer fiktiven Welt. Dabei gibt es eine Story die während des Spielens erlebt wird. Damit diese Geschichte voran geht, muss der Spieler verschiedene Missionen bzw. Quests erledigen, um einen Fortschritt im Spiel zu erlangen. Dabei handelt es sich um die verschiedensten Aufgaben. Darüber hinaus sammelt der Spieler Objekte in der Welt, welche er anschließend im Spiel nutzen kann. Ein Beispiel für solche Spiele ist das Spiel The Witcher 3, welches auf diesen Prinzipien aufbaut. %todo: <-- verlinkung auf wichter 3 like: "für weitere Informationen:..."
	
Darüber hinaus gibt es noch sogenannte MMORPGs, Massive Multiplayer Online Role Player Game. Dabei gibt es wie bei RPGs eine Story und Quests, allerdings kann man auch auf andere Spieler treffen und mit diesen gemeinsam spielen. Das wohl berühmteste MMORPG ist dabei World of Warcraft %todo: auch hier ein link für weitere Informationen
vom Entwickler Blizzard. Hier kann der Spieler zwischen verschiedenen Klassen einen Charakter auswählen, von menschenähnlichen bis zu skurrilen Gestalten. Jede dieser Klassen hat verschiedene Fähigkeiten, welche sich auf den Spielverlauf auswirken. Darüber hinaus gibt es in MMORPGs verschiedene Events, an denen die Spieler gemeinsam versuchen eine Quest zu erfüllen. Dieser Multiplayermodus grenzt die MMORPGs von den RPG ab.

NoRPG hat keine Multiplayer Möglichkeiten und wird deswegen nur als ein RPG klassifiziert. Allerdings muss der Spieler keine Quests erfüllen, wie sie aus anderen Spielen bekannt sind. Der Nutzer muss andere Spiele spielen, damit der Spieler voran schreiten kann. Deshalb wurde sich für den Namen NoRPG entschieden, da es sich bei dem Spiel um ein RPG handelt, aber nicht alle klassischen Eigenschaften eines RPGs besitzt.
	
NoRPG ist ein Spiel, welches versucht Bildung für jeden erreichbar zu machen. Dieses Ziel ist dabei in den Global Goals definiert. Dabei handelt es sich um 17 Ziele welche bis 2030 Umgesetzt werden sollen, um das Leben für alle Menschen auf der Welt zu verbessern\footnote{vgl. Global Goals \cite{global} (2017)}. 2015 haben 193 Weltführer diese Unterzeichnet und begonnen dieses umzusetzen. Dabei sind diese Ziele umfangreich und reichen von einem besseren Umgang mit den uns zur Verfügung stehenden Ressourcen bis hin zu qualitativ hochwertiger Bildung für jeden und kostenlos.

NoRPG unterstützt dabei das Ziel hochwertige Bildung für jeden Zugänglich zu machen, da die App für jeden frei, kostenlos und überall angeboten wird. Dieses Ziel hat weitere Unterziele, wobei nun kurz auf die für NoRPG relevanten Unterziele eingegangen wird.

\begin{itemize}
\item Bis 2030 sicherstellen, dass alle Mädchen und Jungen gleichberechtigt eine kostenlose und hochwertige Grund- und Sekundarschulbildung abschließen, die zu brauchbaren und effektiven Lernergebnissen führt.

Dieses Unterziel wird in NoRPG dahingegen unterstützt, dass die Common Core State Standards implementiert werden. Dies sind Standards für Unterrichtsfächer und beschreiben den zu lernenden Inhalt für Kinder in den verschiedenen Klassen. Da NoRPG für alle kostenfrei zugänglich ist und Mädchen und Jungen gleichberechtigt sind, werden auch diese zwei Aspekte des Unterziels unterstützt. Die Standards werden dabei von den zu Verfügung gestellten Spielen erfüllt.

\item Aufbau und Weiterentwicklung von Bildungseinrichtungen, die kinder- und behindertengerecht und geschlechtsspezifisch sind und für alle eine sichere, gewaltfreie, integrative und effektive Lernumgebung bieten. %todo: vielleciht 2 sätze :P

Da NoRPG keine Bildungseinrichtung ist wird dieses Unterziel nur bedingt erfüllt. NoRPG bietet Kindern jedoch eine sichere, gewaltfreie, integrative und effektive Lernumgebung, wodurch dieses Ziel allerdings zum Teil erfüllt wird. Darüber hinaus ist diese Umgebung für Kinder ausgelegt. Geschlechtsspezifisch ist das Spiel nur dahingehend, dass die Kinder zu Beginn das Geschlecht ihres Charakters auswählen.

\end{itemize}


	
\section{Gedanken} %todo: anderer Kapitelname: weil es handelt nur von Gamifikation

Dieses Ziel der Global Goals wird dabei durch Gamification umgesetzt. Bei Gamification oder Gamifizierung handelt es sich um das nutzen von spielespezifischen Eigenschaften, welche außerhalb von Spielen angewendet werden. Ein Beispiel dafür ist PayBack. Dabei sammeln die Nutzer bei jedem Einkauf Punkte. Diese Punkte können die Kunden dann gegen Prämien eintauschen. In Videospielen sammeln die Spieler zum Beispiel Münzen um diese anschließend einzutauschen. 

Gamification setzt darüber hinaus auch weitere erfolgreiche Prinzipien aus Videospielen um, damit die Nutzer Spaß daran haben, das Produkt zu nutzen. Dadurch soll die Motivation gesteigert werden, NoRPG zu verwenden. Beispiele für weitere Praktiken die Gamification einsetzten:
\begin{itemize}
\item Einbettung in eine Geschichte
\item direktes Feedback
\item Belohnungen
\item Status durch Level und Auszeichnungen
\item Wettbewerb
\item Teamaktivität
\end{itemize}

Darüber hinaus sind die Faktoren wie Erfolgserlebnisse, Gruppenzugehörigkeit und soziale Akzeptanz wichtig.

	
\section{Story}

In NoRPG spielt der Spieler einen Charaker der zu Begin der Geschichte in einem Dorf wohnt. Von diesem können der Spieler durch Portale in zunächst fünf andere Welten gehen, in denen andere Lebensbedingungen herrschen. Eines Tages wurde dieses Dorf von einem Bösewicht heimgesucht und dieser hat die komplette Welt, in der der Spieler lebt, farblos gemacht. Zusammen mit den Farben wurden die Emotionen aus dem Dorf genommen. Nun hat sich der Hauptcharakter das Ziel gesetzt, die Farben und die Emotionen der Einwohner zurück zu bringen, damit wieder alles wie vor dem Angriff ist. Dazu muss der Spieler in die verschiedenen Welten gehen und verschiedene Aufgaben erfüllen.

Die Welten sind dabei in fünf verschiedene Themengebiete Aufgeteilt, Walt, Tropen, Wüste, Eis und Feuer.
Die Welten sind jeweils wieder in fünf Bereiche gegliedert, die der Spieler mit der Zeit erreichen kann. Dabei wird der zu erkundende Bereich größer, je mehr Standarts der Spieler erfüllt hat. %Eine Exemplarische Aufteilung einer Welt kann in Abbildung \ref{aufteilung_welten} gesehen werden.

%todo: vllt den vorabschnitt so: Die von der Stadt erreichbaren Welt haben ihr eigenes Motto und Thema. Jede Welt ist dabei in Unterbereiche gegliedert, die der Spieler mit der Zeit erreichen kann. Dabei wird der zu erkundende Bereich immer größer, je mehr Standards der Spieler erfüllt. Jede von den Dungeon erreichbare Welten haben das selbe Konzept. Der Spieler kann dort verschiedene Spiele finden, die beispielsweise von einem fahrenden Händler angeboten werden oder in einer Truhe zu finden sind. Darüberhinaus kann der Spieler Sammelgegenstände finden, die für die Story wichtig sind.

%\begin{figure}[htbp]
%\centering 
%\label{aufteilung_welten}
%\includegraphics[width=0.9\textwidth]{pics/aufteilung_welten.png}
%\caption{Exemplarische Aufteilung der Spielwelt}
%\end{figure}

%todo: Ich habe die \textbf in \subsubsection geändert, tauchen nicht im Inhaltsverzeichnis auf

\subsubsection{Startwelt: Dorf}
	In die verschiedenen Welten gelangen kann der Spieler über Portale, welche in der Startwelt vorhanden sind.
	Jede der fünf Welten spiegelt eine Klasse wieder, so sind Standards der ersten Klasse eins in der ersten Welt. Äquivalent verhält es sich mit den anderen Welten.
	
\subsubsection{Welt 1: Wald}
	In dieser Welt ist Wald das primäre Element. Dieser ist in verschiedenen Ausprägungen vorhanden, von sehr Dicht bis hin zu Lichtungen.
	
\subsubsection{Welt 2: Tropen}
	Diese Welt erinnert an die Tropen. Es wird sehr bunt und es wird viel Tropenwald, jedoch auch viel Wasser geben. Diese Welt besteht dabei aus mehreren Inseln, welche über Brücken erreicht werden können.
	
\subsubsection{Welt 3: Wüste}
	In Welt 3 dreht sich alles um die Wüste. Diese besteht aus Dünen und ist trostloser wie die vorherigen Welten. Innerhalb dieser Wüste gibt es Oasen, verschiedene Ruinen und Pyramiden zu entdecken. 
	
\subsubsection{Welt 4: Eis}
	In der Eiswelt dreht sich alles um Eis und Schnee. Es wird viele hohe Berge mit großen Höhlen geben. 
	
\subsubsection{Welt 5: Lava}
	In Welt 5 dreht sich alles um Feuer. In dieser Welt gibt es verschiedene Inselplattformen, welche durch Lava getrennt sind. Diese sind untereinander mit Brücken verbunden. Auf den verschiedenen Inseln kann der Spieler verschiedene Dinge erkunden, darunter Drachen, Vulkane oder Ruinen. 
	
	%todo: vielleicht hier noch bisschen ausführlicher später, wenn wir die Welten fertig haben

%In Welt 1, einer Waldwelt, muss der Spieler verschiedene Edelsteine sammeln und Aufgaben erfüllen, um die ersten Farben zurück zu erlangen. Sobald dies geschehen ist kommt die Farbe Grün in das Dorf des Spielers zurück. Nun muss der Spieler in die anderen Welten gehen und die Farbe zurück holen. Über dem Dorf wird sein aktueller Fortschritt als Regenbogen dargestellt, welcher umso bunter und voller wird, unso mehr Edelsteine gefunden und Aufgaben abgeschlossen wurden. Sobald der Spieler alle Edelsteine gefunden und alle Aufgaben abgeschlossen hat, hat er die Welt gerettet und es kommt eine Endsequenz.
	
\section{Abgrenzung zu MOOC}

NoRPG ist allerdings kein Massive Open Online Course (MOOC) sondern baut nur auf einem auf. Die Daten dieses MOOCs werden in NoRPG zur verfügung gestellt, damit die Kinder einen Anreiz haben, diesen zu nutzen. Der eigentliche MOOC ist dabei Hone. Dabei handelt es sich um eine Lernplattform, welche es für Kinder ermöglicht, grundlegendes Verständniss, welches in der Grundschule vermittelt wird, zu erlangen. Dabei stellt Hone den Kindern eine Liste von Spielen bereit. Durch spielen dieser Spiele lernen die Kinder die Dinge die sie auch in der Grundschule lernen würden. Dabei sind diese Lerninhalte in den CCSS definiert und aufgegliedert.