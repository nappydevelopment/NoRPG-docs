\chapter{NoRPG}		
Bei NoRPG handelt es sich um eine Gamifizierung einer Lernspielplattform. Dabei soll NoRPG an ein Rollenspiel, bzw. Role Player Game (RPG), erinnern und implementiert dessen charakteristische Eigenschaften. Eine dieser Eigenschaften von RPGs ist, dass der Spielende in die Rolle realer Menschen, fiktiver Figuren, Tiere oder auch Gegenstände einer fiktiven Welt schlüpft\footnote{vgl. Warwitz und Rudolf \cite{rpgSinn} Seite 78ff.}. Dabei wird eine Story erzählt, die das Kind erleben kann. Damit der Spieler allerdings in der Geschichte vorankommt, muss er verschiedene Missionen bzw. Quests erledigen. Dabei kann es sich um die verschiedensten Aufgaben handeln. Darüber hinaus sammelt der Spieler Objekte in der Welt, welche er anschließend im Spiel nutzen kann. Ein Beispiel für solche Spiele ist das Spiel The Witcher 3, welches auf diesen Prinzipien aufbaut\footnote{für weitere Informationen siehe \url{http://thewitcher.com/en/witcher3}}.
	
Des Weiteren gibt es noch MMORPGs, Massive Multiplayer Online Role Player Game. Dabei gibt es wie bei RPGs eine Story und Quests, allerdings kann man auch auf andere Spieler treffen und mit ihnen gemeinsam spielen. Das wohl berühmteste MMORPG ist dabei World of Warcraft \footnote{für weitere Informationen siehe \url{https://worldofwarcraft.com/}} vom Entwickler Blizzard. In MMORPGs gibt es die Möglichkeit verschiedene Events mit mehreren Spieler gemeinsam Quests zu erfüllen. Dieser Mehrspieler-Modus grenzt die MMORPGs von den RPGs ab.

Jedoch handelt es sich bei NoRPG letztendlich nicht um ein klassisches RPG oder MMORPG, sondern um eine Lernspielplattform und bietet Lernspiele zum Herunterladen an. NoRPG soll durch die Eigenschaften eines Rollenspiels die Spieler dazu anregen, weitere Lernspiele herunterzuladen und zu spielen. Deshalb wurde sich für den Namen NoRPG entschieden, da nicht alle charakteristischen Eigenschaften implementiert werden.

NoRPG ist allerdings auch nicht mit einem Massive Open Online Course (MOOC) zu verwechseln, sondern baut nur auf einem auf. Ein MOOC bezeichnet einen kostenlosen Onlinekurs. Ein MOOC würde selbst die Lerninhalte anbieten\footnote{Vgl. Porter \cite{moocBook} Seite 3f.}, wohingegen in NoRPG nur Spiele angeboten werden, die den Lerninhalt bereitstellen.

\section{The Global Goals}
NoRPG ist ein Spiel, welches versucht Bildung für jeden erreichbar zu machen. Dieses Ziel ist dabei in den Global Goals definiert. Dabei handelt es sich um 17 Ziele welche bis 2030 Umgesetzt werden sollen, um das Leben für alle Menschen auf der Welt zu verbessern\footnote{vgl. Global Goals \cite{global} (2017)}. 2015 haben 193 Weltführer diese Unterzeichnet und begonnen dieses umzusetzen. Dabei sind diese Ziele umfangreich und reichen von einem besseren Umgang mit den uns zur Verfügung stehenden Ressourcen bis hin zu qualitativ hochwertiger Bildung für jeden und kostenlos.

NoRPG unterstützt dabei das Ziel, hochwertige Bildung für jeden zugänglich zu machen. Dieses Ziel wird beschrieben als Sicherung eines integrierenden Bildungssystems für alle und die Förderung von gleichberechtigten und hochwertigen lebenslangen Lernchancen\footnote{\url{http://www.globalgoals.org/de/global-goals/quality-education/}}. Dieses Ziel hat weitere Unterziele, wobei nun kurz auf die für NoRPG relevanten Unterziele eingegangen wird.

\begin{itemize}
\item Bis 2030 sicherstellen, dass alle Mädchen und Jungen gleichberechtigt eine kostenlose und hochwertige Grund- und Sekundarschulbildung abschließen, die zu brauchbaren und effektiven Lernergebnissen führt.

\item Aufbau und Weiterentwicklung von Bildungseinrichtungen, die kinder- und behindertengerecht und geschlechtsspezifisch sind und für alle eine sichere, gewaltfreie, integrative und effektive Lernumgebung bieten.
\end{itemize}

Das erste Unterziel wird in NoRPG dahingegen unterstützt, dass die Common Core State Standards implementiert werden. Da NoRPG für alle kostenfrei zugänglich ist und Mädchen und Jungen gleichberechtigt sind, werden auch diese zwei Aspekte des Unterziels unterstützt. 

Da NoRPG keine Bildungseinrichtung ist wird das zweite Unterziel nur bedingt erfüllt. NoRPG bietet Kindern jedoch eine sichere, gewaltfreie, integrative und effektive Lernumgebung, wodurch dieses Ziel allerdings zum Teil erfüllt wird. Darüber hinaus kann diese Anwendung in Bildungseinrichtungen angewendet werden. Geschlechtsspezifisch ist das Spiel nur dahingehend, dass die Kinder zu Beginn das Geschlecht ihres Charakters auswählen.

\section{Common Core State Standards}
Die Common Core sind ein Set von hochqualitativen akademischen Standards für Mathe und Englisch. Diese Lernziele skizzieren, was ein Schüler wissen und fähig sein am Ende jeder Klasse sollte. Die Standards wurden geschaffen um sicherzustellen, dass alle Schüler mit den gelernten Fähigkeiten und Kenntnissen im Collage, Karriere und im Leben, egal wo sie leben, erfolgreich zu sein.

Das Problem des Schulsystems ist, dass die Standards von Staat zu Staat variieren und stimmen meistens nicht mit dem überein, was die Kinder wissen sollten. In Erkennung der Notwendigkeit konsequenter Lernziele wurden die Common Core State Standards entwickelt\footnote{Vgl. \url{http://www.corestandards.org/about-the-standards/}}.

Die Standards werden dabei von den zur Verfügung gestellten Spielen erfüllt. NoRPG sorgt dafür, dass diese Standards in der korrekten Reihenfolge und vollständig erfüllt werden. Betrachtet werden zunächst nur die Standards für die ersten fünf Klassen.

\subsubsection{Mathe}
Die Common Core State Standards konzentrieren sich auf eine klare Reihe von mathematischen Fähigkeiten und Konzepten. Die Schüler lernen Konzepte in einer organisierten Weise. Die Standards ermutigen die Schüler, reale Probleme zu lösen\footnote{Vgl. \url{http://www.corestandards.org/Math/}}.

Das Fach Mathe lässt sich für die ersten fünf Klassen in insgesamt fünf Themenbereiche eingliedern. Zu den Themen gehören beispielsweise algebraisches Denken, Operationen im Zahlenraum bis 100, Maßeinheiten und Geometrie. Jedes dieser einzelnen Themen sind nochmals in Standards unterteilt. Erst durch diese genaue Gliederung wird es ermöglicht, dass alle wichtigen Inhalte abgedeckt werden.

\subsubsection{Englisch}
Die Common Core State Standards bittet die Schüler, Geschichten und Literatur zu lesen, sowie komplexere Texte, die Fakten und Hintergrundwissen in Bereichen wie Wissenschaft und Sozialwissenschaften liefern. Die Schüler werden herausgefordert und gefragt, welche sie dazu bringen, auf das zurückzugreifen, was sie gelesen haben. Dies unterstreicht kritisches denken, Problemlösung und analytische Fähigkeiten, die für den Erfolg in College, Karriere und Leben erforderlich sind\footnote{Vgl. \url{http://www.corestandards.org/ELA-Literacy/}}.

Das Fach Englisch lässt sich für die ersten fünf Klassen in insgesamt sechs Themenbereiche eingliedern. Zu den Themen gehört das Lesen, Schreiben, Sprechen und Zuhören sowie Grammatik. Hier gilt es genauso, dass die Themenbereiche nochmals gegliedert werden.

\section{Gamifizierung}
Dieses Ziel der Global Goals soll dabei durch die Gamifizierung der Common Core State Standards umgesetzt werden. Gamifizierung bzw. Gamification bezieht sich auf die Analyse von spielespezifischen Eigenschaften, welche die Spiele unterhaltsam machen und diese dann in Situationen außerhalb von Spielen anzuwenden, um das Gefühl von Spaß für neue Anwendungen, wie Lernen oder Marketing, zu übertragen\footnote{Umformuliert vom Oxford Dictionary}. Ein Beispiel dafür ist PayBack. Dabei sammeln die Nutzer bei jedem Einkauf Punkte. Diese können die Kunden dann gegen Prämien eintauschen. In Videospielen sammeln die Spieler zum Beispiel Münzen um diese anschließend gegen Gegenstände einzutauschen. 

Gamifizierung verwendet darüber hinaus noch weitere erfolgreiche Prinzipien aus Videospielen, um die Nutzer zu Motivieren das Produkt zu nutzen. Beispiele für weitere Prinzipien\footnote{Vgl. Berkling, Faller und Piertzik \cite{gamesPaper} Seite 3}, die Gamifizierung einsetzten:

\begin{itemize}
\item Einbettung in eine Geschichte
\item direktes Feedback
\item Belohnungen
\item Status durch Level und Auszeichnungen
\item Wettbewerb
\item Teamaktivität
\end{itemize}

Darüber hinaus sind die Faktoren wie Erfolgserlebnisse, Gruppenzugehörigkeit und soziale Akzeptanz wichtig. Im nächsten Unterkapitel wird eine das Prinzip, die Einbettung in eine Geschichte, genauer betrachtet und wieso eine Geschichte so wichtig ist.
	
\section{Die Geschichte}
In NoRPG spielt der Spieler einen Charakter, der in einem Dorf wohnt. Von diesem kann der Spieler durch Portale in zunächst fünf verschiedene Welten gehen, in denen andere Lebensbedingungen herrschen. Am Anfang der Geschichte wird dieses Dorf von einem Bösewichten heimgesucht. Dieser hat das komplette Dorf farblos gemacht. Zusammen mit den Farben wurden die Emotionen aus dem Dorf genommen. Nun hat sich der Hauptcharakter, der Spieler, das Ziel gesetzt, die Farben zurück zu bringen. Dazu muss der Spieler in die verschiedenen Welten gehen und verschiedene Quests erfüllen.

Die einzelnen Welten repräsentieren eine Klassenstufe und sind dementsprechend anspruchsvoll und unterschiedlich gestaltet. Die von der Stadt erreichbaren Welt haben ihr eigenes Motto und Thema. Jede Welt ist dabei in Unterbereiche gegliedert, die der Spieler mit der Zeit erreichen kann. Dabei wird der zu erkundende Bereich immer größer, je weiter der Spieler in der Geschichte vorankommt. 

\subsubsection{Startwelt: Das Dorf Rutherglen}
	Rutherglen ist die Heimat des Spielers und dient als Brücke zwischen allen Welten. In die verschiedenen Welten gelangt der Spieler über Portale, die in Rutherglen verteilt sind. Bei dem Heimartort des Spielers handelt es sich um ein verschlagenes, unscheinbares und ruhiges Dorf, dementsprechend auch die Bewohner. Der Spieler kann in dieser Welt mit allen Bewohnern interagieren und sprechen. Diese erzählen Geschichten, geben Tipps oder betreiben Smalltalk.
	
	Die restlichen folgenden Welten haben grundsätzlich das gleiche Konzept. Jede der fünf Welten spiegelt eine Klasse wieder, so sind Standards der ersten Klasse in der ersten Welt. Äquivalent verhält es sich mit den anderen Welten. Neben den Spielen kann der Spieler noch Truhen und weitere Bewohner der Welten treffen und mit ihnen interagieren.
	
\subsubsection{Welt 1: Die dichten Wälder von Talhan}
	In Talhan ist Wald das primäre Element. Dieser ist in verschiedenen Ausprägungen vorhanden, von sehr Dicht bis hin zu Lichtungen, von Laubbäumen, über topische Bäume bis hinzu Fichtenbäume. Die erste von den gestohlenen Farben kann hier wiedergefunden werden. GRÜNE FARBE
	
\subsubsection{Welt 2: Die tropischen Inseln von Galapagos}
	Die tropischen Inseln von Galapagos sind sehr farbenfroh, allerdings gibt es auch viel Wasser. Diese Welt besteht dabei aus mehreren Inseln, welche mit dem Schiff erreicht werden können. Themen sind unter einem Piraten, aber auch das tropische Wachstum. BLAUe Farbe vom Regenbogen
	
\subsubsection{Welt 3: Die endlose Wüste Kalahari}
	Kalahari ist eine sehr große Wüste. Diese besteht aus Dünen und ist trostloser wie die vorherigen Welten. Innerhalb dieser Wüste gibt es Oasen, verschiedene Ruinen und ägyptische Wahrzeichen, beispielsweise Pyramiden, zu entdecken. Gelbe Farbe
	
\subsubsection{Welt 4: Das verschneiten Gebirge Lhotse}
	In der Eiswelt dreht sich alles um Eis und Schnee. Es wird viele hohe Berge mit großen Höhlen geben. Türkis --> Eis
	
\subsubsection{Welt 5: Der Vulkan Ätna}
	In Welt 5 dreht sich alles um Feuer. In dieser Welt gibt es verschiedene Inselplattformen, welche durch Lava getrennt sind. Diese sind untereinander mit Brücken verbunden. Auf den verschiedenen Inseln kann der Spieler verschiedene Dinge erkunden, darunter Drachen, Vulkane oder Ruinen. Rote farbe
