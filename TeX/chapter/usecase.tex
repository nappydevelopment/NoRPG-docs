\chapter{Software Requirements Specification}
	Das Software Requirements Specification, kurz SRS, ist ein veröffentlichter Standard zur Spezifikation von Anforderungen einer Software. Das SRS dient zur Kommunikation zwischen Stakeholders und Entwickler. Das SRS beschreibt den Projektumfang und stellt eine Referenz für Entwickler dar, um Informationen für die Entwickler bereitzustellen.
	
	Dieser Teil der Arbeit stellt das SRS und somit die Anforderungen und Richtlinien für das Spiel dar.
	
\section{Allgemeine Beschreibung}
	NoRPG ist eine Android Plattform für Lernspiele. Allgemein ist die Herausforderung NoRPG interessant für Kinder darzustellen, so dass die Kinder lust haben neue Lernspiel herunterzuladen und dabei noch Spaß haben.
	
	\subsection{Benutzermerkmale}
		Die Kinder sollten Erfahrungen mit der Verwendung eines Smartphones, insbesondere mit einem Android-System, haben. Wissen wie in Google Play Spiele gedownloaded werden. Und Erfahrungen im Lesen und Schreiben für die Registrierung, Anmeldung und durchführen von NPC-Kommunikation zum verstehen des ablaufes der Plattform.
	
	\subsection{Beschränkungen} 
		Zum spielen wird keine Internetverbindung vorausgesetzt. Es gibt Funktionen wie bspw. Synchronisierung, Download oder andere die jedoch eine Internetverbindung voraussetzten.  

\section{Funktionalitäten}
	Use Cases dokumentieren Funktionalitäten eines Systems auf Basis von einfachen Modellen. In einem Use Case wird das nach außen sichtbare Verhalten eines Systems aus der Sicht der Nutzer beschrieben. Ein Nutzer kann hierbei eine Person, eine Rolle oder ein anderes System sein. Dieser Nutzer tritt als Akteur mit dem System in Interaktion, um ein bestimmtes Ziel zu erreichen.
	
	\begin{center}
		\includegraphics[width=\textwidth]{pics/OUCD.pdf}
		\captionof{figure}{Overall Use Case Diagramm} 
	\end{center}
	
	In der Grafik sind 2 Systeme zu sehen. Links das vorhandene System Hone welches ein Frontend für die Entwickler und für die Kinder darstellt. Die Kinder sollen nicht mehr über Hone die Spiele herunterladen sondern nur noch die App NoRPG verwenden. Die Ansicht wird jedoch weiterhin genutzt und soll den Eltern der Kinder die Möglichkeit geben, den Fortschritt des Kindes nachzuschauen. Hone soll von den Rollen Entwickler und Eltern entwickelt werden.
	
	Das rechte System NoRPG stellt die zu entwickelnde App dar. Diese dient als Frontend für das Kind. Es gibt viele Use Cases. Die Use Cases werden in unterschiedliche Gruppen zusammengefasst. Die nächsten Unterkapitel sind die einzelnen Gruppierungen.
	
	\subsection{Login}
		Dieser Use Case beschreibt den Anwendungsfall, dass der Benutzer sich anmelden möchte.
			
		\subsubsection{Ereignisablauf}
			
		\subsubsection{Spezielle Anforderungen}		
			
		\subsubsection{Vorbedingungen}
			
		\subsubsection{Nachbedingungen}
		
	\subsection{Register}
		Dieser Use Case beschreibt den Anwendungsfall, dass der Benutzer 
			
		\subsubsection{Ereignisablauf}
			
		\subsubsection{Spezielle Anforderungen}		
			
		\subsubsection{Vorbedingungen}
			
		\subsubsection{Nachbedingungen}
	
	\subsection{Create character}
		Dieser Use Case beschreibt den Anwendungsfall, dass der Benutzer 
			
		\subsubsection{Ereignisablauf}
			
		\subsubsection{Spezielle Anforderungen}		
			
		\subsubsection{Vorbedingungen}
			
		\subsubsection{Nachbedingungen}
	
	\subsection{Open menu}
		Dieser Use Case beschreibt den Anwendungsfall, dass der Benutzer 
			
		\subsubsection{Ereignisablauf}
			
		\subsubsection{Spezielle Anforderungen}		
			
		\subsubsection{Vorbedingungen}
			
		\subsubsection{Nachbedingungen}
	
	\subsection{Open map}
		Dieser Use Case beschreibt den Anwendungsfall, dass der Benutzer 
			
		\subsubsection{Ereignisablauf}
			
		\subsubsection{Spezielle Anforderungen}		
			
		\subsubsection{Vorbedingungen}
			
		\subsubsection{Nachbedingungen}
	
	\subsection{Show games}
		Dieser Use Case beschreibt den Anwendungsfall, dass der Benutzer 
			
		\subsubsection{Ereignisablauf}
			
		\subsubsection{Spezielle Anforderungen}		
			
		\subsubsection{Vorbedingungen}
			
		\subsubsection{Nachbedingungen}
	
	\subsection{Show colors}
		Dieser Use Case beschreibt den Anwendungsfall, dass der Benutzer 
			
		\subsubsection{Ereignisablauf}
			
		\subsubsection{Spezielle Anforderungen}		
			
		\subsubsection{Vorbedingungen}
			
		\subsubsection{Nachbedingungen}
	
	\subsection{View progress}
		Dieser Use Case beschreibt den Anwendungsfall, dass der Benutzer 
			
		\subsubsection{Ereignisablauf}
			
		\subsubsection{Spezielle Anforderungen}		
			
		\subsubsection{Vorbedingungen}
			
		\subsubsection{Nachbedingungen}
	
	\subsection{View math}
		Dieser Use Case beschreibt den Anwendungsfall, dass der Benutzer 
			
		\subsubsection{Ereignisablauf}
			
		\subsubsection{Spezielle Anforderungen}		
			
		\subsubsection{Vorbedingungen}
			
		\subsubsection{Nachbedingungen}
	
	\subsection{View english}
		Dieser Use Case beschreibt den Anwendungsfall, dass der Benutzer 
			
		\subsubsection{Ereignisablauf}
			
		\subsubsection{Spezielle Anforderungen}		
			
		\subsubsection{Vorbedingungen}
			
		\subsubsection{Nachbedingungen}
	
	\subsection{Account}
		Dieser Use Case beschreibt den Anwendungsfall, dass der Benutzer 
			
		\subsubsection{Ereignisablauf}
			
		\subsubsection{Spezielle Anforderungen}		
			
		\subsubsection{Vorbedingungen}
			
		\subsubsection{Nachbedingungen}
	
	\subsection{Game}
		Dieser Use Case beschreibt den Anwendungsfall, dass der Benutzer 
			
		\subsubsection{Ereignisablauf}
			
		\subsubsection{Spezielle Anforderungen}		
			
		\subsubsection{Vorbedingungen}
			
		\subsubsection{Nachbedingungen}
	
	\subsection{Player Interaction}
		Dieser Use Case beschreibt den Anwendungsfall, dass der Benutzer 
			
		\subsubsection{Ereignisablauf}
			
		\subsubsection{Spezielle Anforderungen}		
			
		\subsubsection{Vorbedingungen}
			
		\subsubsection{Nachbedingungen}
	
	\subsection{NPC Interaction}
		Dieser Use Case beschreibt den Anwendungsfall, dass der Benutzer 
			
		\subsubsection{Ereignisablauf}
			
		\subsubsection{Spezielle Anforderungen}		
			
		\subsubsection{Vorbedingungen}
			
		\subsubsection{Nachbedingungen}
	
	\subsection{Choose Games}
		Dieser Use Case beschreibt den Anwendungsfall, dass der Benutzer 
			
		\subsubsection{Ereignisablauf}
			
		\subsubsection{Spezielle Anforderungen}		
			
		\subsubsection{Vorbedingungen}
			
		\subsubsection{Nachbedingungen}
	
	\subsection{Synchronize}
		Dieser Use Case beschreibt den Anwendungsfall, dass der Benutzer 
			
		\subsubsection{Ereignisablauf}
			
		\subsubsection{Spezielle Anforderungen}		
			
		\subsubsection{Vorbedingungen}
			
		\subsubsection{Nachbedingungen}
	
	\subsection{Save local}
		Dieser Use Case beschreibt den Anwendungsfall, dass der Benutzer 
			
		\subsubsection{Ereignisablauf}
			
		\subsubsection{Spezielle Anforderungen}		
			
		\subsubsection{Vorbedingungen}
			
		\subsubsection{Nachbedingungen}
	
	
	
\section{Benutzbarkeit}
hier steht noch ein bisschen text und so, das und das
	
	\subsection{Anlernzeit}
		asdfasf
	
	\subsection{Hardware Anforderungen}
		braucht zum Downloaden von "NoRPG" und den angebotenen Lernspielen eine Internetverbindung	
	
	\subsection{Software Anforderungen}
		asdsadasd
	
	\subsection{Verfügbarkeit}
		asdasdas
		
	\subsection{Performanz}
	
\section{Schnittstellen}
hgierasoujdas asdjkoiiasjd asdhjukasdjn asm,dhasjkd

	\subsection{Benutzer Schnittstellen}
		MockUps etc.
		
	\subsection{Hardware Schnittstellen}
		Server? Datenbank?		
		
	\subsection{Software Schnittstellen}
		Zu Hone
