\chapter{Software Requirements Specification}
	Das Software Requirements Specification, kurz SRS, ist ein veröffentlichter Standard zur Spezifikation einer Software. Die Struktur eines SRS ist vom Institute of Electrical and Electronics Engineers im Standard IEEE 830-1998 festgehalten.
	
	Der weitere Aufbau dieses Kapitels entspricht dem des Standards.
	
\section{Einführung}
	Dieses Unterkapitel des SRS stellt einen Überblick über das gesamte SRS bereit.
	
	\subsection{Zweck}
		Das SRS beschreibt den Projektumfang und die Anforderungen an die Software. Dabei beschreibt der Verfasser beispielsweise die Funktionalität, die externen Schnittstellen und die Performanz\footnote{vgl. Tripp \cite{srsIEEE}(1998) Seite 3}. 
	
		Die Zielgruppe des SRSs sind zunächst alle, die in irgendeiner Verbindung mit der Software stehen oder jene, die Interesse an der Umsetzung interessiert sind. Zudem dient die Spezifikation zur Kommunikation zwischen Stakeholder und Entwickler.
		
	\subsection{Umfang}
		Dieses SRS handelt von dem Software Produkt NoRPG. Bei NoRPG handelt es sich um eine Gamifizierung einer Lernspielbibliothek für Android. Allgemein stellt NoRPG nur Lernspiele bereit. Jedoch ist NoRPG wie ein Rollenspiel aufgebaut und erfordert das Spielen von Lernspielen, um weitere Spiele freizuschalten. Jedoch ist es kein Rollenspiel im üblichen Sinne.
		
		NoRPG richtet sich an Kinder und soll durch den Aufbau eines Rollenspiels die Benutzer dazu anregen, weitere Spiele herunterzuladen und zu spielen. Im nächsten Kapitel werden unter anderem die Gedanken zur Gestaltung und weitere Ziele von NoRPG genauer betrachtet.
		
	\subsection{Übersicht}
		
\section{Allgemeine Beschreibung}
	In diesem Unterkapitel werden die allgemeinen Faktoren, die das Produkt und seine Anforderungen betreffen, beschrieben. Dieses Kapitel behandelt nicht die spezifischen Anforderungen sondern stellt den Hintergrund für diese Anforderungen dar. 

	\subsection{Produktperspektive}
		In diesem Unterabschnitt des SRS wird das Produkt in unterschiedlichen Perspektiven mit verwandten Produkten betrachtet, denn bei NoRPG handelt es sich nicht um ein selbstständiges abgeschlossenes Projekt.
		
		For each interface:
		Discussion of the purpose of the interfacing software as related to this software product
		
		Definition of the interface in terms of message content and format. It is not necessary to detail any well-documented interface, but a reference to the document defining the interface is required.
		
		\subsubsection{System interfaces}
			Hone ist das Frontend für Spielentwickler und für Benutzer. Spielentwickler können ihre Spiele bei Hone zur Verfügung stellen und die Benutzer können auf der Webplattform unterschiedliche Lernspiele herunterladen.
		
			NoRPG stellt das verbesserte und kindgerechte Frontend der Benutzer bereit. Der Benutzer kann alle Aktionen aus der App steuern. Jedoch können sich die Benutzer weiterhin auf der Webplattform anmelden. 
		
		\subsubsection{User interfaces}
			It should include both content and format as follows:
			
			Name of item; Description of purpose; Source of input or destination of output; Valid range, accuracy, and/or tolerance; Units of measure; Timing; Relationships to other inputs/outputs; Screen formats/organization; Window formats/organization; Data formats; Command formats; End messages.
		
			The logical characteristics of each interface between the software product and its users. This includes those conÞguration characteristics (e.g., required screen formats, page or window layouts, content of any reports or menus, or availability of programmable function keys) necessary to accomplish the software requirements.
			
			All the aspects of optimizing the interface with the person who must use the system. This may simply comprise a list of do's and don'ts on how the system will appear to the user. One example may be a requirement for the option of long or short error messages.
		
		\subsubsection{Hardware interfaces}
			This should specify the logical characteristics of each interface between the software product and the hardware components of the system. This includes conÞguration characteristics (number of ports, instruction sets, etc.).
			
			It also covers such matters as what devices are to be supported, how they are to be supported, and protocols. For example, terminal support may specify full-screen support as opposed to line-by-line support.
		
		\subsubsection{Software interfaces}
			This should specify the use of other required software products (e.g., a data management system, an operating system, or a mathematical package), and interfaces with other application systems (e.g., the linkage between an accounts receivable system and a general ledger system). For each required software product, the following should be provided:Name, Specification number, Version number and Source.
		
		\subsubsection{Memory constraints}
			This should specify any applicable characteristics and limits on primary and secondary memory
		
	\subsection{Produktfunktionen}
		This subsection of the SRS should provide a summary of the major functions that the software will perform. For example, an SRS for an accounting program may use this part to address customer account maintenance, customer statement, and invoice preparation without mentioning the vast amount of detail that each of those functions requires.
		
		Sometimes the function summary that is necessary for this part can be taken directly from the section of the higher-level speciÞcation (if one exists) that allocates particular functions to the software product. Note that for the sake of clarity The functions should be organized in a way that makes the list of functions understandable to the customer or to anyone else reading the document for the first time.
		
		Textual or graphical methods can be used to show the different functions and their relationships. Such a diagram is not intended to show a design of a product, but simply shows the logical relationships among variables.
	
	\subsection{Benutzermerkmale}
		Grundsätzlich richtet sich die App NoRPG an Kinder, die keine Möglichkeit haben eine Schule zu besuchen oder zusätzlich lernen wollen. Jedoch werden keine anderen Benutzergruppen für diese App ausgeschlossen.
		
		Der Benutzer sollte Erfahrung mit der Verwendung eines Smartphones, insbesondere mit Android-Systemen, haben. Dazu zählt die Bedienung der Android-Oberfläche und insbesondere die Bedienung des Google Play Stores.
		
		Da es sich bei den Benutzern in den meisten Fällen um Kinder handelt, sollten diese englische Texte lesen und verstehen können. Denn zum voranschreiten muss der Benutzer die Unterhaltungen mit NPC zum herunterladen von Spielen verstehen können um die richtige Aktion auszuwählen.
	
	\subsection{Einschränkungen} 
		Einschränkungen für den Entwickler
				
		Regulatory policies, Hardware limitations, Interfaces to other applications, Parallel operation, Audit functions, Control functions, Higher-order language requirements, Signal handshake protocols, Reliability requirements, Criticality of the application and Safety and security considerations
	
	\subsection{Annahmen und Abhängigkeiten}
		This subsection of the SRS should list each of the factors that affect the requirements stated in the SRS. These factors are not design constraints on the software but are, rather, any changes to them that can affect the requirements in the SRS. For example, an assumption may be that a speciÞc operating system will be available on the hardware designated for the software product. If, in fact, the operating system is not available, the SRS would then have to change accordingly.

\section{Spezifische Anforderungen}
	This section of the SRS should contain all of the software requirements to a level of detail sufÞcient to enable designers to design a system to satisfy those requirements, and testers to test that the system satisÞes those requirements. Throughout this section, every stated requirement should be externally perceivable by users, operators, or other external systems. These requirements should include at a minimum a description of
every input (stimulus) into the system, every output (response) from the system, and all functions performed by the system in response to an input or in support of an output. As this is often the largest and most important part of the SRS, the following principles apply:

	SpeciÞc requirements should be stated in conformance with all the characteristics described in 4.3.
	
	SpeciÞc requirements should be cross-referenced to earlier documents that relate
	
	All requirements should be uniquely identiÞable
	
	Careful attention should be given to organizing the requirements to maximize readability

	\subsection{Funktionale Anforderungen}
		Use Cases dokumentieren Funktionalitäten eines Systems auf Basis von einfachen Modellen. In einem Use Case wird das nach außen sichtbare Verhalten eines Systems aus der Sicht der Nutzer beschrieben. Ein Nutzer kann hierbei eine Person, eine Rolle oder ein anderes System sein. Dieser Nutzer tritt als Akteur mit dem System in Interaktion, um ein bestimmtes Ziel zu erreichen.
	
		\begin{center}
			\includegraphics[width=\textwidth]{pics/OUCD.pdf}
			\captionof{figure}{Overall Use Case Diagramm} 
		\end{center}
	
		In der Grafik sind 2 Systeme zu sehen. Links das vorhandene System Hone welches ein Frontend für die Entwickler und für die Kinder darstellt. Die Kinder sollen nicht mehr über Hone die Spiele herunterladen sondern nur noch die App NoRPG verwenden. Die Ansicht wird jedoch weiterhin genutzt und soll den Eltern der Kinder die Möglichkeit geben, den Fortschritt des Kindes nachzuschauen. Hone soll von den Rollen Entwickler und Eltern entwickelt werden.
	
		Das rechte System NoRPG stellt die zu entwickelnde App dar. Diese dient als Frontend für das Kind. Es gibt viele Use Cases. Die Use Cases werden in unterschiedliche Gruppen zusammengefasst. Die nächsten Unterkapitel sind die einzelnen Gruppierungen.
	
		\subsubsection{Login}
			Dieser Use Case beschreibt den Anwendungsfall, dass der Benutzer sich bei NoRPG anmelden möchte. Eine Anmeldung ist notwendig um NORPG zu starten. Die Besonderheit bei der Anmeldung  ist, dass der Account des Benutzers auch für die Anmeldung bei Hone benötigt wird. Der Benutzer kann sich in NoRPG im Startbildschirm anmelden und anschließend das Spiel zu starten.
			
			\paragraph{Ereignisablauf}
				Eingeben von Benutzername und Passwort.	Klicke auf Login.
			
				Alternativer Ablauf: Abbruch oder Spiel Beenden
			
			\paragraph{Vorbedingungen}
				Benutzer ist registriert, Während der Anmeldung ist eine Internetverbindung vorhanden, Kombination von Benutzername und Passwort existiert, es ist kein anderer Benutzer auf dem Gerät angemeldet
			
			\paragraph{Nachbedingungen}
				Benutzer angemeldet, kann online oder offline weiterspielen. Beim nächsten Start der App ist der Benutzer automatisch angemeldet.
			
				Oder falls die eingegeben Daten nicht übereinstimmen, Fehlermeldung anzeigen
	
		\subsubsection{Create character}
			Dieser Use Case beschreibt den Anwendungsfall, dass der Benutzer seinen Charakter erstellen möchte. Dies ist eine einmalige Aktion, die beim ersten Anmelden durchlaufen wird. 
			
			\paragraph{Ereignisablauf}
				Wenn der Benutzer sich zum ersten mal anmeldet hat er die Möglichkeit seinen Charakter zu erstellen. Dafür wählt der Benutzer sich zunächst sein Geschlecht aus und wählt anschließend den passenden Charakter.
			
				Zum Abschluss vergibt der Benutzer seinen Charakter einen Namen.
	
			\paragraph{Vorbedingungen}
				Der Account meldet sich das erste mal in der App an.
			
			\paragraph{Nachbedingungen}
				Nach der Erstellung beginnt das Spiel und der Charakter ist gespeichert. Bei erneuter Anmeldung muss der Benutzer nicht erneut einen Charakter erstellen.
	
		\subsubsection{Player interaction}
			Dieser Use Case beschreibt den Anwendungsfall: Benutzer interaktionen, wie Bewegen oder Bestätigen.
			
			\paragraph{Ereignisablauf}
				Klickt auf Pfeiltasten, Charakter bewegt sich in diese Richtung
			
				Klickt auf A, Charakter bestätigt
			
				Klickt auf B, Charakter lehnt ab
			
				(Bild Mockup)
			
			\paragraph{Vorbedingungen}
				Spieler befindet sich im Spiel (nicht loading screen und menü ist geschlossen)
			
			\paragraph{Nachbedingungen}
				Charakter bewegt sich, bestätigt oder lehnt ab
	
		\subsubsection{NPC interaction}
			Dieser Use Case beschreibt den Anwendungsfall, dass der Benutzer sich in einer Interaktion mit einem NPC befindet. NPC bedeutet Non-Player Charakter und stellt die programmierten Charaktere dar (Unterhaltungen mit NPC, Storytelling)
			
			\paragraph{Ereignisablauf}
				Ereignisablauf etc.

			\paragraph{Vorbedingungen}
				Ingame, nicht loading screen oder menü offen
			
			\paragraph{Nachbedingungen}
				Unterhaltung findet statt, etc.
	
		\subsubsection{Choose games}
			Dieser Use Case beschreibt den Anwendungsfall, dass der Benutzer ein spiel zum downloaden auswählt
			
			\paragraph{Ereignisablauf}
				Der Benutzer kann sich (wenn vorhanden) zwischen mehrere Spielen auswählen um den Kurs abzuschließen.

			\paragraph{Vorbedingungen}
				Internetverbindung, darf die SPiele nach dem Standard spielen
			
			\paragraph{Nachbedingungen}
				Weiterleitung auf Google Play Store

		\subsubsection{Open map}
			Dieser Use Case beschreibt den Anwendungsfall, dass der Benutzer die Karte öffnet. Die Karte dient zur Orientierung der Welt und beinhaltet Symbole etc. um herauszufinden was so ist
			
			\paragraph{Ereignisablauf}
			Benutzer öffnet Menü und klickt auf "Map" ...
	
			\paragraph{Vorbedingungen}
				Menü offen, Benutzer befindet sich nicht in einer NPC Interaktion
			
			\paragraph{Nachbedingungen}
				Eine Karte von der aktuellen Welt wird geöffnet
	
		\subsubsection{Show games}
			Dieser Use Case beschreibt den Anwendungsfall: Liste der gespielten und heruntergeladneen Spiele wird angezeigt. Zuordnung zu den Standards. Aus NoRPG das Spiel starten können.
			
			\paragraph{Ereignisablauf}
				Benuter öffnet Menü und klickt auf "Games" ...

			\paragraph{Vorbedingungen}
				Menü offen, Benutzer befindet sich nicht in einer NPC interaktion
			
			\paragraph{Nachbedingungen}
				Eine Liste wird angezeigt
	
		\subsubsection{View progress}
			Dieser Use Case beschreibt den Anwendungsfall, dass der Benutzer 
			
			\paragraph{Ereignisablauf}
	
			\paragraph{Vorbedingungen}
			
			\paragraph{Nachbedingungen}
	
		\subsubsection{Settings}
			Dieser Use Case beschreibt den Anwendungsfall, dass der Benutzer 
			
			\paragraph{Ereignisablauf}
	
			\paragraph{Vorbedingungen}
			
			\paragraph{Nachbedingungen}
	
		\subsubsection{Synchronize}
			Dieser Use Case beschreibt den Anwendungsfall, dass der Benutzer 
			
			\paragraph{Ereignisablauf}
	
			\paragraph{Vorbedingungen}
			
			\paragraph{Nachbedingungen}
	
		\subsubsection{Save local}
			Dieser Use Case beschreibt den Anwendungsfall, dass der Benutzer 
			
			\paragraph{Ereignisablauf}
	
			\paragraph{Vorbedingungen}
			
			\paragraph{Nachbedingungen}
	
	\subsection{Performanz Anforderungen}
		This subsection should specify both the static and the dynamic numerical requirements placed on the software or on human interaction with the software as a whole. Static numerical requirements may include the following
		
		The number of terminals to be supported, The number of simultaneous users to be supported, Amount and type of information to be handled
		
	\subsection{Datenbank Anforderungen}
		This should specify the logical requirements for any information that is to be placed into a database. This may include the following:
		
		Types of information used by various functions
		
		Frequency of use
		
		Accessing capabilities
		
		Data entities and their relationships
		
		Integrity constraints
		
		Data retention requirements.
	
	\subsection{Entwurfsbeschränkungen}
		This should specify design constraints that can be imposed by other standards, hardware limitations, etc.
		
		Standards compliance: This subsection should specify the requirements derived from existing standards or regulations. They may include the following: 
		
		Report format, Data naming, Accounting procedures and Audit tracing.
	
	\subsection{Benuzterfreundlichkeit}

	\subsection{Zuverlässigkeit}
		This should specify the factors required to establish the required reliability of the software system at time of delivery.
	
	\subsection{Verfügbarkeit}
		This should specify the factors required to guarantee a deÞned availability level for the entire system such as checkpoint, recovery, and restart. 
	
	\subsection{Sicherheit}
		This should specify the factors that protect the software from accidental or malicious access, use, modiÞcation, destruction, or disclosure. SpeciÞc requirements in this area could include the need to
		
		Utilize certain cryptographical techniques; 
		
		Keep speciÞc log or history data sets;
		
		Assign certain functions to different modules;
		
		Restrict communications between some areas of the program;
		
		Check data integrity for critical variables.
	
	\subsection{Wartbarkeit}
		This should specify attributes of software that relate to the ease of maintenance of the software itself. There may be some requirement for certain modularity, interfaces, complexity, etc. Requirements should not be placed here just because they are thought to be good design practices.
	
	\subsection{Portabilität}
		This should specify attributes of software that relate to the ease of porting the software to other host machines and/or operating systems. This may include the following:
		
		Percentage of components with host-dependent code;
		
		Percentage of code that is host dependent;
		
		Use of a proven portable language; 
		
		Use of a particular compiler or language subset;
		
		Use of a particular operating system.
