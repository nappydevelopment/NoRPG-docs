\chapter{Software Requirements Specification}
	Das Software Requirements Specification, kurz SRS, ist ein veröffentlichter Standard zur Spezifikation von Anforderungen einer Software. Das SRS beschreibt den Projektumfang und dient zur Kommunikation zwischen Stakeholders und Entwickler. Stakeholders können im SRS ihre Anforderungen für die Software beschreiben und die Entwickler können aus dem SRS Informationen für die Implementierung gewinnen.
	
	Das SRS hat eine feste Struktur, jedoch muss die Spezifikation einer Software nicht alle Punkte beinhalten. Die folgenden Kapitel stellen die Kapitel des SRS dar.
	
\section{Allgemeine Beschreibung}
	NoRPG ist eine spielbasierte Lernspielplattform für Android. NoRPG selbst ist wie ein Rollenspiel aufgebaut, jedoch stellt es kein Rollenspiel dar. In NoRPG werden unterschiedliche Lernspiele zum herunterladen angeboten.
	
	NoRPG unterstützt die Global Goals, 4. Ziel Quality Education. Paar Worte dazu verlieren.		
	
	Weitere Informationen in Kapitel 4 über die Gedanken für die Story, Design etc.
	
	\subsection{Benutzermerkmale}
		Die Kinder sollten Erfahrungen mit der Verwendung eines Smartphones, insbesondere mit einem Android-System, haben. Wissen wie in Google Play Spiele gedownloaded werden. Und Erfahrungen im Lesen und Schreiben für die Registrierung, Anmeldung und durchführen von NPC-Kommunikation zum verstehen des ablaufes der Plattform.
	
	\subsection{Beschränkungen} 
		Zum spielen wird keine Internetverbindung vorausgesetzt. Es gibt Funktionen wie bspw. Synchronisierung, Download oder andere die jedoch eine Internetverbindung voraussetzten.  

\section{Funktionalitäten}
	Use Cases dokumentieren Funktionalitäten eines Systems auf Basis von einfachen Modellen. In einem Use Case wird das nach außen sichtbare Verhalten eines Systems aus der Sicht der Nutzer beschrieben. Ein Nutzer kann hierbei eine Person, eine Rolle oder ein anderes System sein. Dieser Nutzer tritt als Akteur mit dem System in Interaktion, um ein bestimmtes Ziel zu erreichen.
	
	\begin{center}
		\includegraphics[width=\textwidth]{pics/OUCD.pdf}
		\captionof{figure}{Overall Use Case Diagramm} 
	\end{center}
	
	In der Grafik sind 2 Systeme zu sehen. Links das vorhandene System Hone welches ein Frontend für die Entwickler und für die Kinder darstellt. Die Kinder sollen nicht mehr über Hone die Spiele herunterladen sondern nur noch die App NoRPG verwenden. Die Ansicht wird jedoch weiterhin genutzt und soll den Eltern der Kinder die Möglichkeit geben, den Fortschritt des Kindes nachzuschauen. Hone soll von den Rollen Entwickler und Eltern entwickelt werden.
	
	Das rechte System NoRPG stellt die zu entwickelnde App dar. Diese dient als Frontend für das Kind. Es gibt viele Use Cases. Die Use Cases werden in unterschiedliche Gruppen zusammengefasst. Die nächsten Unterkapitel sind die einzelnen Gruppierungen.
	
	\subsection{Login}
		Dieser Use Case beschreibt den Anwendungsfall, dass der Benutzer sich bei NoRPG anmelden möchte. Eine Anmeldung ist notwendig um NoRPG zu starten. 
			
		\subsubsection{Ereignisablauf}
			Eingeben von Benutzername und Passwort.	Klicke auf Login.
			
			Alternativer Ablauf: Abbruch
			
		\subsubsection{Vorbedingungen}
			Benutzer ist registriert, Während der Anmeldung ist eine Internetverbindung vorhanden, Kombination von Benutzername und Passwort existiert
			
		\subsubsection{Nachbedingungen}
			Benutzer angemeldet, kann online oder offline weiterspieleny
	
	\subsection{Create character}
		Dieser Use Case beschreibt den Anwendungsfall, dass der Benutzer 
			
		\subsubsection{Ereignisablauf}
	
		\subsubsection{Vorbedingungen}
			
		\subsubsection{Nachbedingungen}
	
	\subsection{Player interaction}
		Dieser Use Case beschreibt den Anwendungsfall, dass der Benutzer 
			
		\subsubsection{Ereignisablauf}
			
		\subsubsection{Vorbedingungen}
			
		\subsubsection{Nachbedingungen}
	
	\subsection{NPC interaction}
		Dieser Use Case beschreibt den Anwendungsfall, dass der Benutzer 
			
		\subsubsection{Ereignisablauf}

		\subsubsection{Vorbedingungen}
			
		\subsubsection{Nachbedingungen}
	
	\subsection{Choose games}
		Dieser Use Case beschreibt den Anwendungsfall, dass der Benutzer 
			
		\subsubsection{Ereignisablauf}

		\subsubsection{Vorbedingungen}
			
		\subsubsection{Nachbedingungen}

	\subsection{Open map}
		Dieser Use Case beschreibt den Anwendungsfall, dass der Benutzer 
			
		\subsubsection{Ereignisablauf}
	
		\subsubsection{Vorbedingungen}
			
		\subsubsection{Nachbedingungen}
	
	\subsection{Show games}
		Dieser Use Case beschreibt den Anwendungsfall, dass der Benutzer 
			
		\subsubsection{Ereignisablauf}

		\subsubsection{Vorbedingungen}
			
		\subsubsection{Nachbedingungen}
	
	\subsection{Show colors}
		Dieser Use Case beschreibt den Anwendungsfall, dass der Benutzer 
			
		\subsubsection{Ereignisablauf}

		\subsubsection{Vorbedingungen}
			
		\subsubsection{Nachbedingungen}
	
	\subsection{View progress}
		Dieser Use Case beschreibt den Anwendungsfall, dass der Benutzer 
			
		\subsubsection{Ereignisablauf}
	
		\subsubsection{Vorbedingungen}
			
		\subsubsection{Nachbedingungen}
	
	\subsection{Settings}
		Dieser Use Case beschreibt den Anwendungsfall, dass der Benutzer 
			
		\subsubsection{Ereignisablauf}
	
		\subsubsection{Vorbedingungen}
			
		\subsubsection{Nachbedingungen}
	
	\subsection{Synchronize}
		Dieser Use Case beschreibt den Anwendungsfall, dass der Benutzer 
			
		\subsubsection{Ereignisablauf}
			
		\subsubsection{Vorbedingungen}
			
		\subsubsection{Nachbedingungen}
	
	\subsection{Save local}
		Dieser Use Case beschreibt den Anwendungsfall, dass der Benutzer 
			
		\subsubsection{Ereignisablauf}
			
		\subsubsection{Vorbedingungen}
			
		\subsubsection{Nachbedingungen}	
	
\section{Benutzbarkeit}
hier steht noch ein bisschen text und so, das und das
	
	\subsection{Anlernzeit}
		asdfasf
	
	\subsection{Hardware Anforderungen}
		keine Ahnung
	
	\subsection{Software Anforderungen}
		Android 4.4 mind
		
\section{Zuverlässigkeit}
	
	\subsection{Verfügbarkeit}
		Offlien verfügbar etc. etc.
		
	\subsection{Performanz}
		Response Time 
	
\section{Schnittstellen}
hgierasoujdas asdjkoiiasjd asdhjukasdjn asm,dhasjkd

	\subsection{Benutzer Schnittstellen}
		MockUps etc.
		
	\subsection{Hardware Schnittstellen}
		Server? Datenbank?		
		
	\subsection{Software Schnittstellen}
		Zu Hone
