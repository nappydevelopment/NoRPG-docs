\chapter{Software Requirements Specification}
	Die Software Requirements Specification, kurz SRS, ist ein veröffentlichter Standard zur Spezifikation von Anforderungen einer Software. Das SRS dient zur Kommunikation zwischen Stakeholders und Entwickler. Die Ziele eines SRS ist die Beschreibung des Projektumfangs und Providing a reference to developers.
	
	In den nächsten Unterkapiteln werden die Use Cases, die Anforderungen sowie die Schnittstellen beschrieben.

\section{Funktionalitäten}
	Use Cases dokumentieren Funktionalitäten eines Systems auf Basis von einfachen Modellen. In einem Use Case wird das nach außen sichtbare Verhalten eines Systems aus der Sicht der Nutzer beschrieben. Ein Nutzer kann hierbei eine Person, eine Rolle oder ein anderes System sein. Dieser Nutzer tritt als Akteur mit dem System in Interaktion, um ein bestimmtes Ziel zu erreichen.
	
	\begin{center}
		\includegraphics[width=\textwidth]{pics/OUCD.pdf}
		\captionof{figure}{Overall Use Case Diagramm} 
	\end{center}
	
	Dieses Projekt hat zehn Use Cases, also Funktionalitäten die der Nutzer ausführen kann. In der Grafik sind 2 Systeme zu sehen. Links das vorhandene System "Hone" welches ein Frontend für die Entwickler und für die Kinder darstellt. Die Kinder sollen nicht mehr über "Hone" die Spiele herunterladen sondern nur noch die App "NoRPG" verwenden. Die Ansicht wird jedoch weiterhin genutzt und soll den Eltern der Kinder die Möglichkeit geben, den Fortschritt des Kindes nachzuschauen. "Hone" soll von den Rollen Entwickler und Eltern entwickelt werden.
	
	Das rechte System "NoRPG" stellt die zu entwickelnde App dar. Diese dient als Frontend für das Kind. *insert User characteristics" --> der User muss englisch/deutsch lesen können, 
	NoRPG nochmals in einen Bereich unterteilt. Dieser Bereich "new child frontend" 
	
	
	\subsection*{Login}
		hier hin kommt die Use Case beschreibung
		
		Vor und Nachbedingung (Auslöser und Ziel)
		
	\subsection*{Synchronize}
		hier hin kommt die Use Case Beschreibung
		
		Vor und Nachbedingung (Auslöser und Ziel)		
			
	\subsection*{Save Local}
		hier hin kommt die Use Case Beschreibung
		
		Vor und Nachbedingung (Auslöser und Ziel)	
	
	\subsection*{Play Offline}
		hier hin kommt die Use Case Beschreibung
		
		Vor und Nachbedingung (Auslöser und Ziel)	
	
\section{Benutzbarkeit}
hier steht noch ein bisschen text und so, das und das
	
	\subsection{Anlernzeit}
		asdfasf
	
	\subsection{Hardware Anforderungen}
		braucht zum Downloaden von "NoRPG" und den angebotenen Lernspielen eine Internetverbindung	
	
	\subsection{Software Anforderungen}
		asdsadasd
	
	\subsection{Verfügbarkeit}
		asdasdas
		
	\subsection{Performanz}
	
\section{Schnittstellen}
hgierasoujdas asdjkoiiasjd asdhjukasdjn asm,dhasjkd

	\subsection{Benutzer Schnittstellen}
		MockUps etc.
		
	\subsection{Hardware Schnittstellen}
		Server? Datenbank?		
		
	\subsection{Software Schnittstellen}
		Zu Hone
