\chapter{Einleitung}

Über Spiele zu spieleplattform und darüber auf den lernende Effekt oder über lernen auf multimediales lernen zu einer Plattform zum lernen bestehend aus einem spiel

Spiele sind ein Bestandteil unserer Kultur schon seit tausenden Jahren. Das erste Spiel soll das Königliches Spiel von Ur gewesen sein, welches bereits 2600 vor Christus existierte. Spiele haben sich seit dem jedoch weiterentwickelt und dienen heutzutage zum munteren Zeitvertreib. Ob als Brett, Karten oder Glückspiel, Spiele sind überall zu finden und jeder kann sie spielen. Seit 1972 entwickeln sich darüber hinaus weitere Spiele, Videospiele. Sie nutzen die immer größer werdende Rechenleistung von Computern aus, um uns immer realistisch aussehender Spiele zu liefern. 
Um den Überblick über die Vielzahl an Videospielen zu behalten, haben sich in den letzten Jahren verschiedene Plattformen etabliert, die versuchen dem Nutzer das zu bieten, was sie suchen. Dabei bieten diese viele verschiedene Arten von Spielen an, die einen beim Spielen die Zeit vergessen lassen.
Allerdings können Spiele uns nicht nur die Zeit vergessen lassen und für heitere Stunden sorgen, sie können uns auch wissen vermitteln. Sei es durch eine Geschichte die sich real abgespielt hat, wie der erste Weltkrieg, oder anderes. Dieses Wissen wird vermittelt unterbewusst an den Nutzer vermittelt, ohne das er aktiv versucht dieses zu lernen.
Für diesen Zweig hat sich eine eigene Branche entwickelt, welche sich mit Lernspielen befasst und versucht uns, über Videospiele, diese Wissen zu vermitteln. Diese Spiele werden hauptsächlich in den Schulen eingesetzt um den Kindern wissen spielerisch zu vermitteln. Jedoch profitiert nicht jedes Kind von diesem Vorteil. Sei es, weil die Schule keine Computer hat, oder weil das Kind nicht zur Schule gehen kann. Für diesen Zweck wurde die Plattform Hone entwickelt, mit der Kinder, die nicht zur Schule gehen können, die Möglichkeit haben, wissen zu erlangen. 

