\chapter{Einleitung}

%Über Spiele zu Spielplattform und darüber auf den lernende Effekt oder über lernen auf multimediales lernen zu einer Plattform zum lernen bestehend aus einem spiel

Spiele, jeder kennt sie und spielt regelmäßig welche. Sie sind ein fester Bestandteil unserer Kultur und das schon seit tausenden Jahren. Die ersten Gesellschaftsspiele wurden noch im Sand mit Stöcken oder Steinen gespielt. Eines der frühesten Spiele wird auch heutzutage noch gespielt, dabei handelt es sich um Mühle, ein Spiel welches die Ägypter vor bereits 3000 Jahren gespielt haben.\footnote{vgl. gesellschaftsspiele.de \cite{spiele} (2015)} Spiele haben sich seit dem jedoch weiterentwickelt und dienen heutzutage nicht nur zum munteren Zeitvertreib. 

Ob als Brett, Karten oder Glücksspiel, Spiele sind überall zu finden und jeder kann sie spielen. Seit 1972 entwickeln sich darüber hinaus weitere Spiele, Videospiele. Sie nutzen die immer größer werdende Rechenleistung von Computern aus, um uns immer realistisch aussehender Spiele zu liefern. 
Um den Überblick über die Vielzahl an Videospielen zu behalten, haben sich in den letzten Jahren verschiedene Plattformen etabliert, die versuchen dem Nutzer das zu bieten, was sie suchen. Dabei stellen diese viele verschiedene Arten von Spielen für die Gamer bereit, die einen beim Spielen die Zeit vergessen lassen. Beispiele für solche Plattformen sind unter anderem Steam, Uplay oder Origin.

Allerdings können Spiele uns nicht nur die Zeit vergessen lassen und für heitere Stunden sorgen, sie können uns auch Wissen vermitteln. Sei es beispielsweise durch eine Geschichte die sich real abgespielt hat, wie der erste Weltkrieg. Dieses Wissen wird unterbewusst an den Nutzer vermittelt, ohne das er aktiv versucht dieses zu lernen.

Für diesen Zweig hat sich eine eigene Branche entwickelt, welche sich mit Lernspielen befasst und versucht uns, über Videospiele, neues Wissen zu vermitteln. Viele dieser Spiele nutzen bekannte Figuren, welche die Kinder aus dem Fernsehen kennen, um dieses Wissen zu vermitteln.

Diese Spiele werden hauptsächlich in den Schulen eingesetzt, um den Kindern wissen spielerisch zu vermitteln. Jedoch profitiert nicht jedes Kind von diesem Vorteil. Sei es, weil die Schule keine Computer hat oder weil das Kind nicht eine Schule besuchen kann. Für diesen Zweck wurde die Plattform Hone\footnote{siehe \url{http://hone-kids.herokuapp.com/}} entwickelt, mit der Kinder, die nicht zur Schule gehen können, die Möglichkeit haben, Wissen zu erlangen. 

\section{Motivation}
Bei Hone handelt es sich um eine Spielplattform auf der sich Kinder, bevorzugt aus Regionen in denen Bildung mangelhaft ist, anmelden können. Auf dieser Web-Plattform haben Kinder dann eine Ansicht Ihrer, durch Spiele gelernte Kompetenzen, und können dort neue Spiele herunterladen, um weitere Kompetenzen zu erwerben. Neben dem Nachteil, dass für die Verwendung ein Computer verwendet werden muss, ist das Aussehen und die Bedienung dieser Plattform allerdings nicht reizvoll für Kinder gestaltet. 

Mehr Kinder als je zuvor in der Geschichte arbeiten täglich mit immer neueren und besseren technischen Geräten. Deshalb soll eine mobile Applikation, kurz App, entwickelt werden, in der die Kinder auf spielerischer weise Fortschritte machen. Durch die Umsetzung als App wird den Kindern eine Plattform angeboten, welche sie unabhängig von dem Standort und der Internetverbindung nutzen können. Ein weiterer Vorteil ist, dass die Kinder den technischen Umgang mit Smartphones lernen.

Die App wird unabhängig von Hone funktionieren und es werden keine Inhalte und Funktionen übernommen.

\section{Ziel der Arbeit}

Die Arbeit hat als Ziel eine App zu gestalten, mit der Kinder spielerisch neues Wissen erwerben können und so fortschritte machen. 

Das Ziel dieser Arbeit ist es dabei eine Dokumentation über die Vorgehensweise zu liefern, sowie eine Dokumentation, mit deren Hilfe andere arbeiten Können.

\section{Aufbau der Arbeit}

Die einzelnen Kapitel dieser Arbeit repräsentieren die notwendigen Schritte, das Ziel zu erreichen.

Bevor spezifiziert werden kann, welche Funktionen eine App haben muss, muss zunächst die Idee für eine App beschrieben werden. Das nächste Kapitel behandelt deshalb die grundlegenden Ideen und Gedanken hinter der App.

Im darauf folgendem Kapitel 3 wird in einem Software Requirements Specifikation die App spezifiziert. Neben funktionalen Anforderungen werden hier auch nicht-funktionale Anforderungen beschrieben. 

Anschließend wird auf die technischen Grundlagen für die Umsetzung eingegangen. Es werden die verwendeten Technologien und der Grund für die Verwendung dieser beschrieben. 

Darauf aufbauend wird die Implementierung erläutert und es wird auf die wesentlichen Stellen in der Umsetzung eingegangen. Abgeschlossen wird diese Arbeit mit einem Fazit und Ausblick, in dem der weitere Werdegang des Projektes geschildert wird.