\chapter{Einleitung}

Über Spiele zu spieleplattform und darüber auf den lernende Effekt oder über lernen auf multimediales lernen zu einer Plattform zum lernen bestehend aus einem spiel

Spiele, jeder kennt sie und spielt regelmäßig welche. Sie sind ein fester bestandteil unserer Kultur und das schon seit tausenden Jahren. Die ersten Gesellschaftsspiele wurden noch im Sand mit Stöcken oder Steinen gespielt.  Eines der frühesten Spiele wird auch heutzutage noch gespielt, dabei handelt es sich um Mühle, ein Spiel welches die Ägypter vor bereits 3000 Jahren gespielt haben.(http://www.gesellschaftsspiele.de/geschichte-brettspiele/) Spiele haben sich seit dem jedoch weiterentwickelt und dienen heutzutage nicht nur zum munteren Zeitvertreib. 

Ob als Brett, Karten oder Glückspiel, Spiele sind überall zu finden und jeder kann sie spielen. Seit 1972 entwickeln sich darüber hinaus weitere Spiele, Videospiele. Sie nutzen die immer größer werdende Rechenleistung von Computern aus, um uns immer realistisch aussehender Spiele zu liefern. 
Um den Überblick über die Vielzahl an Videospielen zu behalten, haben sich in den letzten Jahren verschiedene Plattformen etabliert, die versuchen dem Nutzer das zu bieten, was sie suchen. Dabei bieten diese viele verschiedene Arten von Spielen an, die einen beim Spielen die Zeit vergessen lassen.
Allerdings können Spiele uns nicht nur die Zeit vergessen lassen und für heitere Stunden sorgen, sie können uns auch wissen vermitteln. Sei es durch eine Geschichte die sich real abgespielt hat, wie der erste Weltkrieg, oder anderes. Dieses Wissen wird unterbewusst an den Nutzer vermittelt, ohne das er aktiv versucht dieses zu lernen.

Für diesen Zweig hat sich eine eigene Branche entwickelt, welche sich mit Lernspielen befasst und versucht uns, über Videospiele, diese Wissen zu vermitteln. Viele dieser Spiele nutzen bekannte Figuren, welche die Kinder aus dem Fernseh kennen, um dieses Wissen zu vermitteln.

Diese Spiele werden hauptsächlich in den Schulen eingesetzt um den Kindern wissen spielerisch zu vermitteln. Jedoch profitiert nicht jedes Kind von diesem Vorteil. Sei es, weil die Schule keine Computer hat, oder weil das Kind nicht zur Schule gehen kann. Für diesen Zweck wurde die Plattform Hone entwickelt, mit der Kinder, die nicht zur Schule gehen können, die Möglichkeit haben, wissen zu erlangen. 

\section{Motivation}
Bei Hone handelt es sich um eine Spieleplattform auf der sich Kinder, bevorzugt aus Regionen in denen Bildung mangelhaft ist, anmelden können. Auf dieser Plattform haben Sie dann eine Ansicht Ihrer, durch Spiele gelernte Kompetenzen, bzw. können sich die Kinder dort neue Spiele herunterladen, um weitere Kompetenzen zu erwerben. Das Aussehen und die Bedienung dieser Plattform ist allerdings nicht reizvoll für Kinder gestaltet. 

Deshalb soll eine Spieleapp entwickelt werden, in der die Kinder auf spielerischer weise fortschritte machen. Dabei werden die Inhalte von Hone nicht übernommen und die App funktioniert unabhängig davon. Durch die Umsetzung als App gelingt es darüber hinaus den Kindern eine Offlineplattform zu geben, welche sie unabhängig von der Internetverbindung nutzen können. 

\section{Aufbau der Arbeit}

Die Arbeit beginnt mit der grundlegenden Idee Hinter der App. Darauf folgend wird die spezifikation durchgeführt und dargestzellt. Anschließend wird auf die technischen Grundlagen eingegangen, in denen die verwendeteten Technologien und auf die umzusetzenden Ziele vorgestellt werden, damit verstanden werden kann was getan wurde. Darauf aufbauend wird die Implementierung der zuvor erklärten Ziele erläutert und es wird auf schwirige Stellen in der Umsetzung eingegangen. Abgeschlossen wird diese Arbeit mit einem Fazit und einem kurzen Ausblick, in dem der weitere Werdegang des Projektes geschildert wird.

\section{Ziel der Arbeit}

Die Arbeit hat als Ziel eine App zu gestallten, welche es Kindern spielerisch möglich macht, Lernfortschritte zu erzielen. Die App wird dabei für Smartphones mit dem Betriebssytem Android optimiert.

Das Ziel dieser Arbeit ist es dabei eine Dokumentation über die Vorgehensweise zu liefern, sowie eine Dokumentation, mit deren Hilfe andere arbeiten Können.