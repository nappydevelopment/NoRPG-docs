%
% Header die benutzt werden sollen
%
\documentclass[
  pdftex,
  fontsize=12pt,          % Schriftgroesse
  DIV10,                  % Angabe bzgl Bestimmung der Seitenabstaende
  ngerman,                % fuer Umlaute, Silbentrennung etc.
  paper=a4,               % Papierformat
  twoside=false,          % einseitiges Dokument
  titlepage,              % es wird eine Titelseite verwendet
  parskip=half,           % Abstand zwischen Absaetzen (halbe Zeile)
  headings=normal,        % Groesse der Ueberschriften verkleinern
  listof=nochaptergap,  % Verzeichnisse im Inhaltsverzeichnis auffuehren.
  bibliography=totoc, % Literaturverzeichnis im Inhaltsverzeichnis auffuehren
  index=totoc,            % Index im Inhaltsverzeichnis auffuehren
  captions=tableheading,  % Beschriftung von Tabellen oberhalb ausgeben
  bookmarksopen,
  bookmarksnumbered,
  final                 % Status des Dokuments (final/draft)
]{scrreprt} 

\usepackage[T1]{fontenc}
\usepackage[utf8]{inputenc}
\usepackage[ngerman]{babel}
\usepackage{palatino}

% Zeilenabstand
\usepackage[onehalfspacing]{setspace}

% Code
\usepackage{listings}
\lstdefinestyle{sharpc}{float,
	language=[sharp]C, 
	frame=single,  
	keywordstyle=\bfseries\color{green!40!black},
	commentstyle=\itshape\color{purple!40!black},
	identifierstyle=\color{blue},
	stringstyle=\color{orange}, 
	showstringspaces=false, 
	showspaces=false, 
	numbers=left, 
	captionpos=b, 
	belowcaptionskip=4pt,
	basicstyle=\ttfamily}

%Tabellen
\usepackage{array}
\usepackage{tabularx}
\usepackage{supertabular}
\usepackage{longtable}

%Grafiken
\usepackage[pdftex]{graphicx}
\usepackage{wrapfig}
\usepackage[svgnames]{xcolor}
\usepackage{float}
\usepackage{pdfpages}

%Sprache und Anführungszeichen
\usepackage[ngerman]{babel}
\usepackage[babel,german=quotes]{csquotes}
\usepackage{eurosym}

%Abstände
\usepackage[left=2.5cm, right=2.5cm, top=2.5cm, bottom=2.5cm ]{geometry}

% Pakete um Textteile drehen zu können, oder eine Seite Querformat anzeigen kann.
\usepackage{rotating}
\usepackage{lscape}

%Literaturverweise
%\usepackage{biblatex}
%\usepackage[round]{natbib}
%\usepackage{cite}
\usepackage[style=numeric, backend=biber, citestyle=numeric, dashed=false]{biblatex}

% Hurenkinder und Schusterjungen verhindern
% http://projekte.dante.de/DanteFAQ/Silbentrennung
\clubpenalty=10000
\widowpenalty=10000
\displaywidowpenalty=10000

%Abkürzungsverzeichnis
\usepackage[printonlyused]{acronym}

% Fussnoten
\usepackage[hang, multiple, stable]{footmisc}

\usepackage[%
%	pdftitle={\pdftitel},
%	pdfauthor={\autor},
%	pdfsubject={\arbeit},
%	pdfcreator={pdflatex, LaTeX with KOMA-Script},
%	pdfpagemode=UseOutlines, % Beim Oeffnen Inhaltsverzeichnis anzeigen
%	pdfdisplaydoctitle=true, % Dokumenttitel statt Dateiname anzeigen.
%	pdflang=de % Sprache des Dokuments.
]{hyperref}
\usepackage[all]{hypcap}


% Nummerierung ohne Kapitel-Nr.
\usepackage{chngcntr}
\counterwithout{figure}{chapter}
\counterwithout{equation}{chapter}
\counterwithout{footnote}{chapter}
\counterwithout{table}{chapter}
%
% EOF
%