%\thispagestyle{empty}

\chapter*{Abstract}
	This thesis deals with the implementation of a gamification learning plattform. The goal is the same as the goal from quality education from the global goals. It describes how to secure an integrated education system for all and to promote equal and high-quality lifelong learning opportunities. Through the implementation as a game, the children should be stimulated to have fun at learning.
	
	However, to ensure that the learning process is structured, this goal is supported by the high quality academic standards for Math and English of Common Core State Standards. These outline the learning objectives a student should know and be capable of at the end of each class.
	
	In the first half of this thesis, in addition to the theretical foundations are still all functional and non-functional requirements for implementation in  a software requirements sepecification.
	
	The second half describes the implementation of particularly important elements. In addition to the technical architecture the user interface is treated here and evaluated using a remotetest. The last part describes specific challenges and their implementation. This includes, for example, the implementation of the complete system for the interactions.
	
	Result
	
	
\pagebreak

\chapter*{Zusammenfassung}
	Die vorliegende Arbeit befasst sich mit der Implementierung einer gamifizierten Lernspielplattform. Das Ziel entspricht dabei dem Ziel hochwertige Bildung von den Global Goals. Diese beschreibt die Sicherung eines integrierenden Bildungssystems für alle und die Förderung von gleichberechtigten und hochwertigen lebenslangen Lernchancen. Durch die Implementierung als ein Spiel sollen die Kinder angeregt werden Spaß am Lernen zu haben.
	
	Damit allerdings das Lernen strukturiert stattfindet, wird dieses Ziel dabei durch die hochqualitativen akademischen Standards für Mathe und Englisch von Common Core State Standards unterstützt. Diese skizzieren Lernziele, welche ein Schüler am Ende jeder Klasse wissen und fähig sein sollte.
	
	In der ersten Hälfte dieser Arbeit werden neben den theoretischen Grundlagen noch alle funktionalen und nicht-funktionalen Anforderungen für die Implementierung in einem Software Requirements Specifikation.
	
	In der zweiten Hälfte wird die Implementierung von besonders wichtigen Elementen beschrieben. Neben der technischen Architektur wird hier die Benutzeroberfläche behandelt und anhand einer durchgeführten Remotetest evaluiert. Im letzten Teil werden besondere Herausforderungen und deren Umsetzung beschrieben. Dazu zählt beispielsweise die Implementierung des vollständigen Systems für die Interaktionen.
	
	ERGEBNIS BESCHREIBEN