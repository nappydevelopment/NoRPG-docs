%\thispagestyle{empty}

\chapter*{Abstract}
	This thesis deals with the implementation of a game-based learning game platform. The goal of this work corresponds to the fourth goal of the Global Goals quality education. It describes how to secure an integrated education system for all and to promote equal and high-quality lifelong learning opportunities. Through the implementation as a game, the children should be stimulated to have fun at learning.
	
	However, to ensure that the learning process is structured, this goal is supported by the high-quality academic Common Core State standards for math and English. These outline the learning objectives a student should know and be capable of at the end of each class.
	
	In the first half of this thesis, in addition to the theoretical foundations, all functional and non-functional requirements for the planned implementation are defined in a software requirements specification.
	
	The implementation of NoRPG is described in the second half. In addition to the architecture and its components, the implementation of the user interface is also addressed and evaluated. The evaluation is carried out using a remote usability test. In the last part special components and their implementation are described. This includes, for example, the implementation of the complete system for the interactions.
	
	The result of this work is an app, which is a good step in the right direction, to use learning games more meaningfully and to animate children an effective learning.
	
	
\pagebreak

\chapter*{Zusammenfassung}
	Die vorliegende Arbeit befasst sich mit der Implementierung einer gamifizierten Lernspielplattform. Das Ziel dieser Arbeit entspricht dabei dem vierten Ziel der Global Goals hochwertigen Bildung. Diese beschreibt die Sicherung eines integrierenden Bildungssystems für alle und die Förderung von gleichberechtigten und hochwertigen lebenslangen Lernchancen. Durch die Implementierung als ein Spiel sollen die Kinder angeregt werden, Spaß am Lernen zu haben.
	
	Damit allerdings das Lernen strukturiert stattfindet, wird dieses Ziel dabei durch die hochqualitativen akademischen Common Core State Standards für Mathe und Englisch unterstützt. Diese skizzieren Lernziele, welche ein Schüler am Ende einer jeden Klasse wissen und fähig sein sollte.
	
	In der ersten Hälfte dieser Arbeit werden neben den theoretischen Grundlagen alle funktionalen und nicht-funktionalen Anforderungen für die geplante Umsetzung in einem Software Requirements Specifikation definiert.
	
	In der zweiten Hälfte wird die Implementierung von NoRPG beschrieben. Neben der Architektur und deren Komponenten wird zudem die Umsetzung der Benutzeroberfläche behandelt und evaluiert. Die Evaluation wird mit Hilfe eines Remote Usability Tests durchgeführt. Im letzten Teil werden besondere Komponenten und deren Umsetzungen beschrieben. Dazu zählt beispielsweise die Implementierung des vollständigen Systems für die Interaktionen.
	
	Das Ergebnis dieser Arbeit ist eine App, die einen guten Schritt in die richtige Richtung ist, um Lernspiele sinnvoller zu nutzen und Kinder effektiv zu animieren etwas lernen zu wollen. 